\documentclass[11pt]{article}

    \usepackage[breakable]{tcolorbox}
    \usepackage{parskip} % Stop auto-indenting (to mimic markdown behaviour)
    
    \usepackage{iftex}
    \ifPDFTeX
    	\usepackage[T1]{fontenc}
    	\usepackage{mathpazo}
    \else
    	\usepackage{fontspec}
    \fi

    % Basic figure setup, for now with no caption control since it's done
    % automatically by Pandoc (which extracts ![](path) syntax from Markdown).
    \usepackage{graphicx}
    % Maintain compatibility with old templates. Remove in nbconvert 6.0
    \let\Oldincludegraphics\includegraphics
    % Ensure that by default, figures have no caption (until we provide a
    % proper Figure object with a Caption API and a way to capture that
    % in the conversion process - todo).
    \usepackage{caption}
    \DeclareCaptionFormat{nocaption}{}
    \captionsetup{format=nocaption,aboveskip=0pt,belowskip=0pt}

    \usepackage[Export]{adjustbox} % Used to constrain images to a maximum size
    \adjustboxset{max size={0.9\linewidth}{0.9\paperheight}}
    \usepackage{float}
    \floatplacement{figure}{H} % forces figures to be placed at the correct location
    \usepackage{xcolor} % Allow colors to be defined
    \usepackage{enumerate} % Needed for markdown enumerations to work
    \usepackage{geometry} % Used to adjust the document margins
    \usepackage{amsmath} % Equations
    \usepackage{amssymb} % Equations
    \usepackage{textcomp} % defines textquotesingle
    % Hack from http://tex.stackexchange.com/a/47451/13684:
    \AtBeginDocument{%
        \def\PYZsq{\textquotesingle}% Upright quotes in Pygmentized code
    }
    \usepackage{upquote} % Upright quotes for verbatim code
    \usepackage{eurosym} % defines \euro
    \usepackage[mathletters]{ucs} % Extended unicode (utf-8) support
    \usepackage{fancyvrb} % verbatim replacement that allows latex
    \usepackage{grffile} % extends the file name processing of package graphics 
                         % to support a larger range
    \makeatletter % fix for grffile with XeLaTeX
    \def\Gread@@xetex#1{%
      \IfFileExists{"\Gin@base".bb}%
      {\Gread@eps{\Gin@base.bb}}%
      {\Gread@@xetex@aux#1}%
    }
    \makeatother

    % The hyperref package gives us a pdf with properly built
    % internal navigation ('pdf bookmarks' for the table of contents,
    % internal cross-reference links, web links for URLs, etc.)
    \usepackage{hyperref}
    % The default LaTeX title has an obnoxious amount of whitespace. By default,
    % titling removes some of it. It also provides customization options.
    \usepackage{titling}
    \usepackage{longtable} % longtable support required by pandoc >1.10
    \usepackage{booktabs}  % table support for pandoc > 1.12.2
    \usepackage[inline]{enumitem} % IRkernel/repr support (it uses the enumerate* environment)
    \usepackage[normalem]{ulem} % ulem is needed to support strikethroughs (\sout)
                                % normalem makes italics be italics, not underlines
    \usepackage{mathrsfs}
    

    
    % Colors for the hyperref package
    \definecolor{urlcolor}{rgb}{0,.145,.698}
    \definecolor{linkcolor}{rgb}{.71,0.21,0.01}
    \definecolor{citecolor}{rgb}{.12,.54,.11}

    % ANSI colors
    \definecolor{ansi-black}{HTML}{3E424D}
    \definecolor{ansi-black-intense}{HTML}{282C36}
    \definecolor{ansi-red}{HTML}{E75C58}
    \definecolor{ansi-red-intense}{HTML}{B22B31}
    \definecolor{ansi-green}{HTML}{00A250}
    \definecolor{ansi-green-intense}{HTML}{007427}
    \definecolor{ansi-yellow}{HTML}{DDB62B}
    \definecolor{ansi-yellow-intense}{HTML}{B27D12}
    \definecolor{ansi-blue}{HTML}{208FFB}
    \definecolor{ansi-blue-intense}{HTML}{0065CA}
    \definecolor{ansi-magenta}{HTML}{D160C4}
    \definecolor{ansi-magenta-intense}{HTML}{A03196}
    \definecolor{ansi-cyan}{HTML}{60C6C8}
    \definecolor{ansi-cyan-intense}{HTML}{258F8F}
    \definecolor{ansi-white}{HTML}{C5C1B4}
    \definecolor{ansi-white-intense}{HTML}{A1A6B2}
    \definecolor{ansi-default-inverse-fg}{HTML}{FFFFFF}
    \definecolor{ansi-default-inverse-bg}{HTML}{000000}

    % commands and environments needed by pandoc snippets
    % extracted from the output of `pandoc -s`
    \providecommand{\tightlist}{%
      \setlength{\itemsep}{0pt}\setlength{\parskip}{0pt}}
    \DefineVerbatimEnvironment{Highlighting}{Verbatim}{commandchars=\\\{\}}
    % Add ',fontsize=\small' for more characters per line
    \newenvironment{Shaded}{}{}
    \newcommand{\KeywordTok}[1]{\textcolor[rgb]{0.00,0.44,0.13}{\textbf{{#1}}}}
    \newcommand{\DataTypeTok}[1]{\textcolor[rgb]{0.56,0.13,0.00}{{#1}}}
    \newcommand{\DecValTok}[1]{\textcolor[rgb]{0.25,0.63,0.44}{{#1}}}
    \newcommand{\BaseNTok}[1]{\textcolor[rgb]{0.25,0.63,0.44}{{#1}}}
    \newcommand{\FloatTok}[1]{\textcolor[rgb]{0.25,0.63,0.44}{{#1}}}
    \newcommand{\CharTok}[1]{\textcolor[rgb]{0.25,0.44,0.63}{{#1}}}
    \newcommand{\StringTok}[1]{\textcolor[rgb]{0.25,0.44,0.63}{{#1}}}
    \newcommand{\CommentTok}[1]{\textcolor[rgb]{0.38,0.63,0.69}{\textit{{#1}}}}
    \newcommand{\OtherTok}[1]{\textcolor[rgb]{0.00,0.44,0.13}{{#1}}}
    \newcommand{\AlertTok}[1]{\textcolor[rgb]{1.00,0.00,0.00}{\textbf{{#1}}}}
    \newcommand{\FunctionTok}[1]{\textcolor[rgb]{0.02,0.16,0.49}{{#1}}}
    \newcommand{\RegionMarkerTok}[1]{{#1}}
    \newcommand{\ErrorTok}[1]{\textcolor[rgb]{1.00,0.00,0.00}{\textbf{{#1}}}}
    \newcommand{\NormalTok}[1]{{#1}}
    
    % Additional commands for more recent versions of Pandoc
    \newcommand{\ConstantTok}[1]{\textcolor[rgb]{0.53,0.00,0.00}{{#1}}}
    \newcommand{\SpecialCharTok}[1]{\textcolor[rgb]{0.25,0.44,0.63}{{#1}}}
    \newcommand{\VerbatimStringTok}[1]{\textcolor[rgb]{0.25,0.44,0.63}{{#1}}}
    \newcommand{\SpecialStringTok}[1]{\textcolor[rgb]{0.73,0.40,0.53}{{#1}}}
    \newcommand{\ImportTok}[1]{{#1}}
    \newcommand{\DocumentationTok}[1]{\textcolor[rgb]{0.73,0.13,0.13}{\textit{{#1}}}}
    \newcommand{\AnnotationTok}[1]{\textcolor[rgb]{0.38,0.63,0.69}{\textbf{\textit{{#1}}}}}
    \newcommand{\CommentVarTok}[1]{\textcolor[rgb]{0.38,0.63,0.69}{\textbf{\textit{{#1}}}}}
    \newcommand{\VariableTok}[1]{\textcolor[rgb]{0.10,0.09,0.49}{{#1}}}
    \newcommand{\ControlFlowTok}[1]{\textcolor[rgb]{0.00,0.44,0.13}{\textbf{{#1}}}}
    \newcommand{\OperatorTok}[1]{\textcolor[rgb]{0.40,0.40,0.40}{{#1}}}
    \newcommand{\BuiltInTok}[1]{{#1}}
    \newcommand{\ExtensionTok}[1]{{#1}}
    \newcommand{\PreprocessorTok}[1]{\textcolor[rgb]{0.74,0.48,0.00}{{#1}}}
    \newcommand{\AttributeTok}[1]{\textcolor[rgb]{0.49,0.56,0.16}{{#1}}}
    \newcommand{\InformationTok}[1]{\textcolor[rgb]{0.38,0.63,0.69}{\textbf{\textit{{#1}}}}}
    \newcommand{\WarningTok}[1]{\textcolor[rgb]{0.38,0.63,0.69}{\textbf{\textit{{#1}}}}}
    
    
    % Define a nice break command that doesn't care if a line doesn't already
    % exist.
    \def\br{\hspace*{\fill} \\* }
    % Math Jax compatibility definitions
    \def\gt{>}
    \def\lt{<}
    \let\Oldtex\TeX
    \let\Oldlatex\LaTeX
    \renewcommand{\TeX}{\textrm{\Oldtex}}
    \renewcommand{\LaTeX}{\textrm{\Oldlatex}}
    % Document parameters
    % Document title
    \title{Analiza Listy Przebojów Trójki}
    
    
    
    
    
% Pygments definitions
\makeatletter
\def\PY@reset{\let\PY@it=\relax \let\PY@bf=\relax%
    \let\PY@ul=\relax \let\PY@tc=\relax%
    \let\PY@bc=\relax \let\PY@ff=\relax}
\def\PY@tok#1{\csname PY@tok@#1\endcsname}
\def\PY@toks#1+{\ifx\relax#1\empty\else%
    \PY@tok{#1}\expandafter\PY@toks\fi}
\def\PY@do#1{\PY@bc{\PY@tc{\PY@ul{%
    \PY@it{\PY@bf{\PY@ff{#1}}}}}}}
\def\PY#1#2{\PY@reset\PY@toks#1+\relax+\PY@do{#2}}

\expandafter\def\csname PY@tok@w\endcsname{\def\PY@tc##1{\textcolor[rgb]{0.73,0.73,0.73}{##1}}}
\expandafter\def\csname PY@tok@c\endcsname{\let\PY@it=\textit\def\PY@tc##1{\textcolor[rgb]{0.25,0.50,0.50}{##1}}}
\expandafter\def\csname PY@tok@cp\endcsname{\def\PY@tc##1{\textcolor[rgb]{0.74,0.48,0.00}{##1}}}
\expandafter\def\csname PY@tok@k\endcsname{\let\PY@bf=\textbf\def\PY@tc##1{\textcolor[rgb]{0.00,0.50,0.00}{##1}}}
\expandafter\def\csname PY@tok@kp\endcsname{\def\PY@tc##1{\textcolor[rgb]{0.00,0.50,0.00}{##1}}}
\expandafter\def\csname PY@tok@kt\endcsname{\def\PY@tc##1{\textcolor[rgb]{0.69,0.00,0.25}{##1}}}
\expandafter\def\csname PY@tok@o\endcsname{\def\PY@tc##1{\textcolor[rgb]{0.40,0.40,0.40}{##1}}}
\expandafter\def\csname PY@tok@ow\endcsname{\let\PY@bf=\textbf\def\PY@tc##1{\textcolor[rgb]{0.67,0.13,1.00}{##1}}}
\expandafter\def\csname PY@tok@nb\endcsname{\def\PY@tc##1{\textcolor[rgb]{0.00,0.50,0.00}{##1}}}
\expandafter\def\csname PY@tok@nf\endcsname{\def\PY@tc##1{\textcolor[rgb]{0.00,0.00,1.00}{##1}}}
\expandafter\def\csname PY@tok@nc\endcsname{\let\PY@bf=\textbf\def\PY@tc##1{\textcolor[rgb]{0.00,0.00,1.00}{##1}}}
\expandafter\def\csname PY@tok@nn\endcsname{\let\PY@bf=\textbf\def\PY@tc##1{\textcolor[rgb]{0.00,0.00,1.00}{##1}}}
\expandafter\def\csname PY@tok@ne\endcsname{\let\PY@bf=\textbf\def\PY@tc##1{\textcolor[rgb]{0.82,0.25,0.23}{##1}}}
\expandafter\def\csname PY@tok@nv\endcsname{\def\PY@tc##1{\textcolor[rgb]{0.10,0.09,0.49}{##1}}}
\expandafter\def\csname PY@tok@no\endcsname{\def\PY@tc##1{\textcolor[rgb]{0.53,0.00,0.00}{##1}}}
\expandafter\def\csname PY@tok@nl\endcsname{\def\PY@tc##1{\textcolor[rgb]{0.63,0.63,0.00}{##1}}}
\expandafter\def\csname PY@tok@ni\endcsname{\let\PY@bf=\textbf\def\PY@tc##1{\textcolor[rgb]{0.60,0.60,0.60}{##1}}}
\expandafter\def\csname PY@tok@na\endcsname{\def\PY@tc##1{\textcolor[rgb]{0.49,0.56,0.16}{##1}}}
\expandafter\def\csname PY@tok@nt\endcsname{\let\PY@bf=\textbf\def\PY@tc##1{\textcolor[rgb]{0.00,0.50,0.00}{##1}}}
\expandafter\def\csname PY@tok@nd\endcsname{\def\PY@tc##1{\textcolor[rgb]{0.67,0.13,1.00}{##1}}}
\expandafter\def\csname PY@tok@s\endcsname{\def\PY@tc##1{\textcolor[rgb]{0.73,0.13,0.13}{##1}}}
\expandafter\def\csname PY@tok@sd\endcsname{\let\PY@it=\textit\def\PY@tc##1{\textcolor[rgb]{0.73,0.13,0.13}{##1}}}
\expandafter\def\csname PY@tok@si\endcsname{\let\PY@bf=\textbf\def\PY@tc##1{\textcolor[rgb]{0.73,0.40,0.53}{##1}}}
\expandafter\def\csname PY@tok@se\endcsname{\let\PY@bf=\textbf\def\PY@tc##1{\textcolor[rgb]{0.73,0.40,0.13}{##1}}}
\expandafter\def\csname PY@tok@sr\endcsname{\def\PY@tc##1{\textcolor[rgb]{0.73,0.40,0.53}{##1}}}
\expandafter\def\csname PY@tok@ss\endcsname{\def\PY@tc##1{\textcolor[rgb]{0.10,0.09,0.49}{##1}}}
\expandafter\def\csname PY@tok@sx\endcsname{\def\PY@tc##1{\textcolor[rgb]{0.00,0.50,0.00}{##1}}}
\expandafter\def\csname PY@tok@m\endcsname{\def\PY@tc##1{\textcolor[rgb]{0.40,0.40,0.40}{##1}}}
\expandafter\def\csname PY@tok@gh\endcsname{\let\PY@bf=\textbf\def\PY@tc##1{\textcolor[rgb]{0.00,0.00,0.50}{##1}}}
\expandafter\def\csname PY@tok@gu\endcsname{\let\PY@bf=\textbf\def\PY@tc##1{\textcolor[rgb]{0.50,0.00,0.50}{##1}}}
\expandafter\def\csname PY@tok@gd\endcsname{\def\PY@tc##1{\textcolor[rgb]{0.63,0.00,0.00}{##1}}}
\expandafter\def\csname PY@tok@gi\endcsname{\def\PY@tc##1{\textcolor[rgb]{0.00,0.63,0.00}{##1}}}
\expandafter\def\csname PY@tok@gr\endcsname{\def\PY@tc##1{\textcolor[rgb]{1.00,0.00,0.00}{##1}}}
\expandafter\def\csname PY@tok@ge\endcsname{\let\PY@it=\textit}
\expandafter\def\csname PY@tok@gs\endcsname{\let\PY@bf=\textbf}
\expandafter\def\csname PY@tok@gp\endcsname{\let\PY@bf=\textbf\def\PY@tc##1{\textcolor[rgb]{0.00,0.00,0.50}{##1}}}
\expandafter\def\csname PY@tok@go\endcsname{\def\PY@tc##1{\textcolor[rgb]{0.53,0.53,0.53}{##1}}}
\expandafter\def\csname PY@tok@gt\endcsname{\def\PY@tc##1{\textcolor[rgb]{0.00,0.27,0.87}{##1}}}
\expandafter\def\csname PY@tok@err\endcsname{\def\PY@bc##1{\setlength{\fboxsep}{0pt}\fcolorbox[rgb]{1.00,0.00,0.00}{1,1,1}{\strut ##1}}}
\expandafter\def\csname PY@tok@kc\endcsname{\let\PY@bf=\textbf\def\PY@tc##1{\textcolor[rgb]{0.00,0.50,0.00}{##1}}}
\expandafter\def\csname PY@tok@kd\endcsname{\let\PY@bf=\textbf\def\PY@tc##1{\textcolor[rgb]{0.00,0.50,0.00}{##1}}}
\expandafter\def\csname PY@tok@kn\endcsname{\let\PY@bf=\textbf\def\PY@tc##1{\textcolor[rgb]{0.00,0.50,0.00}{##1}}}
\expandafter\def\csname PY@tok@kr\endcsname{\let\PY@bf=\textbf\def\PY@tc##1{\textcolor[rgb]{0.00,0.50,0.00}{##1}}}
\expandafter\def\csname PY@tok@bp\endcsname{\def\PY@tc##1{\textcolor[rgb]{0.00,0.50,0.00}{##1}}}
\expandafter\def\csname PY@tok@fm\endcsname{\def\PY@tc##1{\textcolor[rgb]{0.00,0.00,1.00}{##1}}}
\expandafter\def\csname PY@tok@vc\endcsname{\def\PY@tc##1{\textcolor[rgb]{0.10,0.09,0.49}{##1}}}
\expandafter\def\csname PY@tok@vg\endcsname{\def\PY@tc##1{\textcolor[rgb]{0.10,0.09,0.49}{##1}}}
\expandafter\def\csname PY@tok@vi\endcsname{\def\PY@tc##1{\textcolor[rgb]{0.10,0.09,0.49}{##1}}}
\expandafter\def\csname PY@tok@vm\endcsname{\def\PY@tc##1{\textcolor[rgb]{0.10,0.09,0.49}{##1}}}
\expandafter\def\csname PY@tok@sa\endcsname{\def\PY@tc##1{\textcolor[rgb]{0.73,0.13,0.13}{##1}}}
\expandafter\def\csname PY@tok@sb\endcsname{\def\PY@tc##1{\textcolor[rgb]{0.73,0.13,0.13}{##1}}}
\expandafter\def\csname PY@tok@sc\endcsname{\def\PY@tc##1{\textcolor[rgb]{0.73,0.13,0.13}{##1}}}
\expandafter\def\csname PY@tok@dl\endcsname{\def\PY@tc##1{\textcolor[rgb]{0.73,0.13,0.13}{##1}}}
\expandafter\def\csname PY@tok@s2\endcsname{\def\PY@tc##1{\textcolor[rgb]{0.73,0.13,0.13}{##1}}}
\expandafter\def\csname PY@tok@sh\endcsname{\def\PY@tc##1{\textcolor[rgb]{0.73,0.13,0.13}{##1}}}
\expandafter\def\csname PY@tok@s1\endcsname{\def\PY@tc##1{\textcolor[rgb]{0.73,0.13,0.13}{##1}}}
\expandafter\def\csname PY@tok@mb\endcsname{\def\PY@tc##1{\textcolor[rgb]{0.40,0.40,0.40}{##1}}}
\expandafter\def\csname PY@tok@mf\endcsname{\def\PY@tc##1{\textcolor[rgb]{0.40,0.40,0.40}{##1}}}
\expandafter\def\csname PY@tok@mh\endcsname{\def\PY@tc##1{\textcolor[rgb]{0.40,0.40,0.40}{##1}}}
\expandafter\def\csname PY@tok@mi\endcsname{\def\PY@tc##1{\textcolor[rgb]{0.40,0.40,0.40}{##1}}}
\expandafter\def\csname PY@tok@il\endcsname{\def\PY@tc##1{\textcolor[rgb]{0.40,0.40,0.40}{##1}}}
\expandafter\def\csname PY@tok@mo\endcsname{\def\PY@tc##1{\textcolor[rgb]{0.40,0.40,0.40}{##1}}}
\expandafter\def\csname PY@tok@ch\endcsname{\let\PY@it=\textit\def\PY@tc##1{\textcolor[rgb]{0.25,0.50,0.50}{##1}}}
\expandafter\def\csname PY@tok@cm\endcsname{\let\PY@it=\textit\def\PY@tc##1{\textcolor[rgb]{0.25,0.50,0.50}{##1}}}
\expandafter\def\csname PY@tok@cpf\endcsname{\let\PY@it=\textit\def\PY@tc##1{\textcolor[rgb]{0.25,0.50,0.50}{##1}}}
\expandafter\def\csname PY@tok@c1\endcsname{\let\PY@it=\textit\def\PY@tc##1{\textcolor[rgb]{0.25,0.50,0.50}{##1}}}
\expandafter\def\csname PY@tok@cs\endcsname{\let\PY@it=\textit\def\PY@tc##1{\textcolor[rgb]{0.25,0.50,0.50}{##1}}}

\def\PYZbs{\char`\\}
\def\PYZus{\char`\_}
\def\PYZob{\char`\{}
\def\PYZcb{\char`\}}
\def\PYZca{\char`\^}
\def\PYZam{\char`\&}
\def\PYZlt{\char`\<}
\def\PYZgt{\char`\>}
\def\PYZsh{\char`\#}
\def\PYZpc{\char`\%}
\def\PYZdl{\char`\$}
\def\PYZhy{\char`\-}
\def\PYZsq{\char`\'}
\def\PYZdq{\char`\"}
\def\PYZti{\char`\~}
% for compatibility with earlier versions
\def\PYZat{@}
\def\PYZlb{[}
\def\PYZrb{]}
\makeatother


    % For linebreaks inside Verbatim environment from package fancyvrb. 
    \makeatletter
        \newbox\Wrappedcontinuationbox 
        \newbox\Wrappedvisiblespacebox 
        \newcommand*\Wrappedvisiblespace {\textcolor{red}{\textvisiblespace}} 
        \newcommand*\Wrappedcontinuationsymbol {\textcolor{red}{\llap{\tiny$\m@th\hookrightarrow$}}} 
        \newcommand*\Wrappedcontinuationindent {3ex } 
        \newcommand*\Wrappedafterbreak {\kern\Wrappedcontinuationindent\copy\Wrappedcontinuationbox} 
        % Take advantage of the already applied Pygments mark-up to insert 
        % potential linebreaks for TeX processing. 
        %        {, <, #, %, $, ' and ": go to next line. 
        %        _, }, ^, &, >, - and ~: stay at end of broken line. 
        % Use of \textquotesingle for straight quote. 
        \newcommand*\Wrappedbreaksatspecials {% 
            \def\PYGZus{\discretionary{\char`\_}{\Wrappedafterbreak}{\char`\_}}% 
            \def\PYGZob{\discretionary{}{\Wrappedafterbreak\char`\{}{\char`\{}}% 
            \def\PYGZcb{\discretionary{\char`\}}{\Wrappedafterbreak}{\char`\}}}% 
            \def\PYGZca{\discretionary{\char`\^}{\Wrappedafterbreak}{\char`\^}}% 
            \def\PYGZam{\discretionary{\char`\&}{\Wrappedafterbreak}{\char`\&}}% 
            \def\PYGZlt{\discretionary{}{\Wrappedafterbreak\char`\<}{\char`\<}}% 
            \def\PYGZgt{\discretionary{\char`\>}{\Wrappedafterbreak}{\char`\>}}% 
            \def\PYGZsh{\discretionary{}{\Wrappedafterbreak\char`\#}{\char`\#}}% 
            \def\PYGZpc{\discretionary{}{\Wrappedafterbreak\char`\%}{\char`\%}}% 
            \def\PYGZdl{\discretionary{}{\Wrappedafterbreak\char`\$}{\char`\$}}% 
            \def\PYGZhy{\discretionary{\char`\-}{\Wrappedafterbreak}{\char`\-}}% 
            \def\PYGZsq{\discretionary{}{\Wrappedafterbreak\textquotesingle}{\textquotesingle}}% 
            \def\PYGZdq{\discretionary{}{\Wrappedafterbreak\char`\"}{\char`\"}}% 
            \def\PYGZti{\discretionary{\char`\~}{\Wrappedafterbreak}{\char`\~}}% 
        } 
        % Some characters . , ; ? ! / are not pygmentized. 
        % This macro makes them "active" and they will insert potential linebreaks 
        \newcommand*\Wrappedbreaksatpunct {% 
            \lccode`\~`\.\lowercase{\def~}{\discretionary{\hbox{\char`\.}}{\Wrappedafterbreak}{\hbox{\char`\.}}}% 
            \lccode`\~`\,\lowercase{\def~}{\discretionary{\hbox{\char`\,}}{\Wrappedafterbreak}{\hbox{\char`\,}}}% 
            \lccode`\~`\;\lowercase{\def~}{\discretionary{\hbox{\char`\;}}{\Wrappedafterbreak}{\hbox{\char`\;}}}% 
            \lccode`\~`\:\lowercase{\def~}{\discretionary{\hbox{\char`\:}}{\Wrappedafterbreak}{\hbox{\char`\:}}}% 
            \lccode`\~`\?\lowercase{\def~}{\discretionary{\hbox{\char`\?}}{\Wrappedafterbreak}{\hbox{\char`\?}}}% 
            \lccode`\~`\!\lowercase{\def~}{\discretionary{\hbox{\char`\!}}{\Wrappedafterbreak}{\hbox{\char`\!}}}% 
            \lccode`\~`\/\lowercase{\def~}{\discretionary{\hbox{\char`\/}}{\Wrappedafterbreak}{\hbox{\char`\/}}}% 
            \catcode`\.\active
            \catcode`\,\active 
            \catcode`\;\active
            \catcode`\:\active
            \catcode`\?\active
            \catcode`\!\active
            \catcode`\/\active 
            \lccode`\~`\~ 	
        }
    \makeatother

    \let\OriginalVerbatim=\Verbatim
    \makeatletter
    \renewcommand{\Verbatim}[1][1]{%
        %\parskip\z@skip
        \sbox\Wrappedcontinuationbox {\Wrappedcontinuationsymbol}%
        \sbox\Wrappedvisiblespacebox {\FV@SetupFont\Wrappedvisiblespace}%
        \def\FancyVerbFormatLine ##1{\hsize\linewidth
            \vtop{\raggedright\hyphenpenalty\z@\exhyphenpenalty\z@
                \doublehyphendemerits\z@\finalhyphendemerits\z@
                \strut ##1\strut}%
        }%
        % If the linebreak is at a space, the latter will be displayed as visible
        % space at end of first line, and a continuation symbol starts next line.
        % Stretch/shrink are however usually zero for typewriter font.
        \def\FV@Space {%
            \nobreak\hskip\z@ plus\fontdimen3\font minus\fontdimen4\font
            \discretionary{\copy\Wrappedvisiblespacebox}{\Wrappedafterbreak}
            {\kern\fontdimen2\font}%
        }%
        
        % Allow breaks at special characters using \PYG... macros.
        \Wrappedbreaksatspecials
        % Breaks at punctuation characters . , ; ? ! and / need catcode=\active 	
        \OriginalVerbatim[#1,codes*=\Wrappedbreaksatpunct]%
    }
    \makeatother

    % Exact colors from NB
    \definecolor{incolor}{HTML}{303F9F}
    \definecolor{outcolor}{HTML}{D84315}
    \definecolor{cellborder}{HTML}{CFCFCF}
    \definecolor{cellbackground}{HTML}{F7F7F7}
    
    % prompt
    \makeatletter
    \newcommand{\boxspacing}{\kern\kvtcb@left@rule\kern\kvtcb@boxsep}
    \makeatother
    \newcommand{\prompt}[4]{
        \ttfamily\llap{{\color{#2}[#3]:\hspace{3pt}#4}}\vspace{-\baselineskip}
    }
    

    
    % Prevent overflowing lines due to hard-to-break entities
    \sloppy 
    % Setup hyperref package
    \hypersetup{
      breaklinks=true,  % so long urls are correctly broken across lines
      colorlinks=true,
      urlcolor=urlcolor,
      linkcolor=linkcolor,
      citecolor=citecolor,
      }
    % Slightly bigger margins than the latex defaults
    
    \geometry{verbose,tmargin=1in,bmargin=1in,lmargin=1in,rmargin=1in}
    
    

\begin{document}
    
    \maketitle
    
    

    
    \hypertarget{projekt---rachunek-prawdopodobieux144stwa-i-statystyka}{%
\section{Projekt - Rachunek prawdopodobieństwa i
statystyka}\label{projekt---rachunek-prawdopodobieux144stwa-i-statystyka}}

\hypertarget{analiza-rekorduxf3w-listy-przebojuxf3w-truxf3jki-z-lat-1982-2020}{%
\subsection{Analiza rekordów Listy Przebojów Trójki z lat
1982-2020}\label{analiza-rekorduxf3w-listy-przebojuxf3w-truxf3jki-z-lat-1982-2020}}

\hypertarget{autor-paweux142-kruczkiewicz}{%
\subsubsection{Autor: Paweł
Kruczkiewicz}\label{autor-paweux142-kruczkiewicz}}

\textbf{Data analizy: 15.01.2021 oraz 23.01.2021}

    W niniejszej pracy pragnę przyjrzeć się jednej z najbardziej znanych i z
pewnością najstarszej liście przebojów w Polsce, tj. Liście Przebojów
Trójki. Lista tworzona w Programie Trzecim Polskiego Radia, której
pomysłodawcą, głównym prowadzącym i twórcą był Marek Niedźwiecki, przez
lata kształtowała gusta muzyczne młodych pokoleń polskich słuchaczy.
Niestety w roku 2020, po tzw.
\href{https://pl.wikipedia.org/wiki/Polskie_Radio_Program_III\#Sprawa_pojawienia_si\%C4\%99_piosenki_\%E2\%80\%9ETw\%C3\%B3j_b\%C3\%B3l_jest_lepszy_ni\%C5\%BC_m\%C3\%B3j\%E2\%80\%9D_na_Li\%C5\%9Bcie_Przeboj\%C3\%B3w}{``Aferze
w radiowej Trójce''}, której przyczynkiem było (jak się miało okazać -
ostatnie) - notowanie nr 1998, Lista Przebojów Radiowej Trójki została
``zawieszona''.

Jako wyraz uznania dlawieloletniej pracy redaktorów Trójki jak i przejaw
zwykłej ciekawości na to, jak wygladały gusta ``Trójkowiczów'' przez
wszystkie te lata, chciałbym krótko przeanalizować niemal 2 tysiące
notowań

    \hypertarget{informacje-wstux119pne}{%
\subsection{Informacje wstępne}\label{informacje-wstux119pne}}

\hypertarget{kruxf3tko-o-danych}{%
\subsubsection{Krótko o danych}\label{kruxf3tko-o-danych}}

Dane zostały pobrane ze strony https://www.lp3.pl/, skąd zostały
``zeskrapowane'' z użyciem języka python z pomocą modułu BeautifulSoup4.
Pełny kod (wraz z plikiem ``requirements.txt'') dostępny jest pod tym
\href{https://github.com/pkrucz00/trojkaStats}{linkiem na githubie} w
pliku \texttt{lp3WebScratch}. Zapisuje on dane do pliku CSV, który to
został zaimportowany za pomocą aplikacji SQLiteStudio do bazy
\texttt{listaPrzebojowTrojki.db}

Baza danych składa się z jednej tabeli zawierającej wszystkie rekordy w
podanym formacie:

\begin{longtable}[]{@{}lll@{}}
\toprule
\begin{minipage}[b]{0.30\columnwidth}\raggedright
Nazwa kolumny\strut
\end{minipage} & \begin{minipage}[b]{0.30\columnwidth}\raggedright
Typ Danych\strut
\end{minipage} & \begin{minipage}[b]{0.30\columnwidth}\raggedright
Opis\strut
\end{minipage}\tabularnewline
\midrule
\endhead
\begin{minipage}[t]{0.30\columnwidth}\raggedright
Nr notowania\strut
\end{minipage} & \begin{minipage}[t]{0.30\columnwidth}\raggedright
INTEGER\strut
\end{minipage} & \begin{minipage}[t]{0.30\columnwidth}\raggedright
Numer notowania (w przypadku notowań podwójnych, dla uproszczenia
numerem notowania jest numer wcześniejszy (np. 665/667 -\textgreater{}
665))\strut
\end{minipage}\tabularnewline
\begin{minipage}[t]{0.30\columnwidth}\raggedright
Rok notowania\strut
\end{minipage} & \begin{minipage}[t]{0.30\columnwidth}\raggedright
INTEGER\strut
\end{minipage} & \begin{minipage}[t]{0.30\columnwidth}\raggedright
Rok, w którym odbyło się notowanie\strut
\end{minipage}\tabularnewline
\begin{minipage}[t]{0.30\columnwidth}\raggedright
Pozycja\strut
\end{minipage} & \begin{minipage}[t]{0.30\columnwidth}\raggedright
INTEGER\strut
\end{minipage} & \begin{minipage}[t]{0.30\columnwidth}\raggedright
Pozycja, jaką osiągnął dany utwór\strut
\end{minipage}\tabularnewline
\begin{minipage}[t]{0.30\columnwidth}\raggedright
Tytuł\strut
\end{minipage} & \begin{minipage}[t]{0.30\columnwidth}\raggedright
STRING\strut
\end{minipage} & \begin{minipage}[t]{0.30\columnwidth}\raggedright
Tytuł utworu\strut
\end{minipage}\tabularnewline
\begin{minipage}[t]{0.30\columnwidth}\raggedright
Artysta\strut
\end{minipage} & \begin{minipage}[t]{0.30\columnwidth}\raggedright
STRING\strut
\end{minipage} & \begin{minipage}[t]{0.30\columnwidth}\raggedright
Wykonawca utworu\strut
\end{minipage}\tabularnewline
\begin{minipage}[t]{0.30\columnwidth}\raggedright
Punkty\strut
\end{minipage} & \begin{minipage}[t]{0.30\columnwidth}\raggedright
INTEGER\strut
\end{minipage} & \begin{minipage}[t]{0.30\columnwidth}\raggedright
Punkty, które dostaje utwór za swoją pozycję (ze wzoru 30 - nr pozycji +
1)\strut
\end{minipage}\tabularnewline
\bottomrule
\end{longtable}

Ponieważ liczba utworów w głównym zestawieniu zmieniała się w różnych
latach, postanowiłem ujednolicić (oczyścić) dane - do bazy trafiają
zawsze te utwory, które zajęły w notowaniu pozycję 30. lub wyższą.
Dzięki temu przyporządkowanie punktów jest jednolite (jednak nieco różni
się od stosowanego przez Radiową Trójkę systemu - np. podsumowaniach
rocznych).

W bazie znajduje się 58815 rekordów. Liczba nie jest podzielna przez 30,
ponieważ na listę często trafiały miejsca \emph{ex aequo}

Kontrowersyjne notowanie 1998 zostało tutaj zamieszczone zgodnie z tym,
jak zostało wyemitowane. Nie wzięto pod uwagę notowania 1999

    \hypertarget{wyux142ux105czenie-ostrzeux17ceux144-i-powiadomieux144-dla-czytelnoux15bci}{%
\subsubsection{Wyłączenie ostrzeżeń i powiadomień (dla
czytelności)}\label{wyux142ux105czenie-ostrzeux17ceux144-i-powiadomieux144-dla-czytelnoux15bci}}

    \begin{tcolorbox}[breakable, size=fbox, boxrule=1pt, pad at break*=1mm,colback=cellbackground, colframe=cellborder]
\prompt{In}{incolor}{68}{\boxspacing}
\begin{Verbatim}[commandchars=\\\{\}]
\PY{n+nf}{options}\PY{p}{(}\PY{n}{warn}\PY{o}{=}\PY{l+m}{\PYZhy{}1}\PY{p}{)}
\PY{n+nf}{options}\PY{p}{(}\PY{n}{dplyr.summarise.inform}\PY{o}{=}\PY{n+nb+bp}{F}\PY{p}{)}
\end{Verbatim}
\end{tcolorbox}

    \hypertarget{podux142ux105czenie-bazy-do-jux119zyka-r}{%
\subsubsection{Podłączenie bazy do języka
R}\label{podux142ux105czenie-bazy-do-jux119zyka-r}}

    \begin{tcolorbox}[breakable, size=fbox, boxrule=1pt, pad at break*=1mm,colback=cellbackground, colframe=cellborder]
\prompt{In}{incolor}{66}{\boxspacing}
\begin{Verbatim}[commandchars=\\\{\}]
\PY{c+c1}{\PYZsh{}Zainstaluj paczkę \PYZdq{}RSQLite\PYZdq{} pozwalającą na użycie bazy danych w postaci obiektu w języku R}
\PY{c+c1}{\PYZsh{}NIE uruchamiaj tej komórki, jeżeli już zainstalowałeś tę paczkę}
\PY{n+nf}{install.packages}\PY{p}{(}\PY{l+s}{\PYZdq{}}\PY{l+s}{wesanderson\PYZdq{}}\PY{p}{)}
\PY{n+nf}{install.packages}\PY{p}{(}\PY{l+s}{\PYZdq{}}\PY{l+s}{RSQLite\PYZdq{}}\PY{p}{)}
\PY{n+nf}{install.packages}\PY{p}{(}\PY{l+s}{\PYZdq{}}\PY{l+s}{tidyverse\PYZdq{}}\PY{p}{)}
\end{Verbatim}
\end{tcolorbox}

    \hypertarget{potrzebne-biblioteki}{%
\subsubsection{Potrzebne biblioteki}\label{potrzebne-biblioteki}}

    \begin{tcolorbox}[breakable, size=fbox, boxrule=1pt, pad at break*=1mm,colback=cellbackground, colframe=cellborder]
\prompt{In}{incolor}{67}{\boxspacing}
\begin{Verbatim}[commandchars=\\\{\}]
\PY{n+nf}{library}\PY{p}{(}\PY{n}{RSQLite}\PY{p}{)}
\PY{n+nf}{library}\PY{p}{(}\PY{n}{tidyverse}\PY{p}{)}
\PY{n+nf}{library}\PY{p}{(}\PY{n}{wesanderson}\PY{p}{)}
\end{Verbatim}
\end{tcolorbox}

    \hypertarget{wypisanie-fragmentu-bazy-danych}{%
\subsubsection{Wypisanie fragmentu bazy
danych}\label{wypisanie-fragmentu-bazy-danych}}

    \begin{tcolorbox}[breakable, size=fbox, boxrule=1pt, pad at break*=1mm,colback=cellbackground, colframe=cellborder]
\prompt{In}{incolor}{69}{\boxspacing}
\begin{Verbatim}[commandchars=\\\{\}]
\PY{n}{con} \PY{o}{\PYZlt{}\PYZhy{}} \PY{n+nf}{dbConnect}\PY{p}{(}\PY{n}{RSQLite}\PY{o}{::}\PY{n+nf}{SQLite}\PY{p}{(}\PY{p}{)}\PY{p}{,} \PY{n}{dbname}\PY{o}{=}\PY{l+s}{\PYZdq{}}\PY{l+s}{ListaPrzebojowTrojki.db\PYZdq{}}\PY{p}{)}

\PY{n+nf}{dbListTables}\PY{p}{(}\PY{n}{con}\PY{p}{)}

\PY{c+c1}{\PYZsh{}Błąd związany z \PYZdq{}mixed type\PYZdq{} związany jest z tytułami i artystami,}
\PY{c+c1}{\PYZsh{}którzy zawierają w swojej nazwie liczby (np. \PYZsq{}51\PYZsq{} zespołu TSA)}
\PY{n}{records} \PY{o}{\PYZlt{}\PYZhy{}} \PY{n+nf}{dbReadTable}\PY{p}{(}\PY{n}{con}\PY{p}{,} \PY{l+s}{\PYZdq{}}\PY{l+s}{Records\PYZdq{}}\PY{p}{)}
\PY{n}{records}\PY{n}{[1}\PY{o}{:}\PY{l+m}{15}\PY{p}{,} \PY{n}{]}

\PY{n+nf}{dbDisconnect}\PY{p}{(}\PY{n}{con}\PY{p}{)}
\end{Verbatim}
\end{tcolorbox}

    'Records'

    
    \begin{tabular}{r|llllll}
 Nr.notowania & Rok.notowania & Pozycja & Tytuł & Artysta & Punkty\\
\hline
	 1                                       & 1982                                    &  1                                      & I'll Find My Way Home                   & Jon \& Vangelis                        & 30                                     \\
	 1                                       & 1982                                    &  2                                      & O! Nie rób tyle hałasu                  & Maanam                                  & 29                                     \\
	 1                                       & 1982                                    &  3                                      & For Those About to Rock (We Salute You) & AC/DC                                   & 28                                     \\
	 1                                       & 1982                                    &  4                                      & 51                                      & TSA                                     & 27                                     \\
	 1                                       & 1982                                    &  5                                      & Opanuj się                              & Perfect                                 & 26                                     \\
	 1                                       & 1982                                    &  6                                      & The Visitors (Crackin' Up)              & ABBA                                    & 25                                     \\
	 1                                       & 1982                                    &  7                                      & Flying Colours                          & Jethro Tull                             & 24                                     \\
	 1                                       & 1982                                    &  8                                      & Pepe wróć                               & Perfect                                 & 23                                     \\
	 1                                       & 1982                                    &  9                                      & Teraz rób co chcesz                     & Budka Suflera                           & 22                                     \\
	 1                                       & 1982                                    & 10                                      & Słodka jest noc                         & Kombi                                   & 21                                     \\
	 1                                       & 1982                                    & 11                                      & Freeze-Frame                            & The J. Geils Band                       & 20                                     \\
	 1                                       & 1982                                    & 12                                      & Przed nami drzwi zamknięte              & Kasa Chorych                            & 19                                     \\
	 1                                       & 1982                                    & 13                                      & O jeden dreszcz                         & Lombard                                 & 18                                     \\
	 1                                       & 1982                                    & 14                                      & Layla                                   & Derek \& the Dominos                   & 17                                     \\
	 1                                       & 1982                                    & 15                                      & Do or Die                               & The Human League                        & 16                                     \\
\end{tabular}


    
    \hypertarget{analiza-wstux119pnaeksploracyjna}{%
\subsection{Analiza
wstępna/eksploracyjna}\label{analiza-wstux119pnaeksploracyjna}}

\hypertarget{poprawnoux15bux107-danych}{%
\subsubsection{Poprawność danych}\label{poprawnoux15bux107-danych}}

Na sam początek możemy sprawdzić, czy zebrane dane są poprawne.
Sprawdźmy liczbę, medianę i średnią wszystkich rekordów

    \begin{tcolorbox}[breakable, size=fbox, boxrule=1pt, pad at break*=1mm,colback=cellbackground, colframe=cellborder]
\prompt{In}{incolor}{6}{\boxspacing}
\begin{Verbatim}[commandchars=\\\{\}]
\PY{n}{records} \PY{o}{\PYZpc{}\PYZgt{}\PYZpc{}} 
\PY{n+nf}{summarise}\PY{p}{(}
    \PY{n}{liczba.rekordów} \PY{o}{=} \PY{n+nf}{n}\PY{p}{(}\PY{p}{)}\PY{p}{,}
    \PY{n}{mediana} \PY{o}{=} \PY{n+nf}{median}\PY{p}{(}\PY{n}{Punkty}\PY{p}{)}\PY{p}{,}
    ś\PY{n}{rednia} \PY{o}{=} \PY{n+nf}{mean}\PY{p}{(}\PY{n}{Punkty}\PY{p}{)}\PY{p}{)}
\end{Verbatim}
\end{tcolorbox}

    \begin{tabular}{r|lll}
 liczba.rekordów & mediana & średnia\\
\hline
	 58815    & 15       & 15.49499\\
\end{tabular}


    
    Jak widać, mediana zbioru to 15, średnia jest bardzo bliska wartości
oczekiwanej. Delikatna różnica spowodowana jest wspomnianymi wcześniej
miejscami ex aequo.

\hypertarget{analiza-pojedynczych-utworuxf3w}{%
\subsubsection{Analiza pojedynczych
utworów}\label{analiza-pojedynczych-utworuxf3w}}

A jak analogiczne statystyki wyglądają dla poszczególnych piosenek?

    \begin{tcolorbox}[breakable, size=fbox, boxrule=1pt, pad at break*=1mm,colback=cellbackground, colframe=cellborder]
\prompt{In}{incolor}{147}{\boxspacing}
\begin{Verbatim}[commandchars=\\\{\}]
\PY{n}{records} \PY{o}{\PYZpc{}\PYZgt{}\PYZpc{}}
    \PY{n+nf}{transmute}\PY{p}{(}
        \PY{n}{Piosenka} \PY{o}{=} \PY{n+nf}{paste}\PY{p}{(}\PY{n}{Tytuł}\PY{p}{,} \PY{n}{Artysta}\PY{p}{,} \PY{n}{sep}\PY{o}{=}\PY{l+s}{\PYZdq{}}\PY{l+s}{ \PYZhy{} \PYZdq{}}\PY{p}{)}\PY{p}{,}
        \PY{n}{Punkty}\PY{o}{=}\PY{n}{Punkty}
    \PY{p}{)} \PY{o}{\PYZpc{}\PYZgt{}\PYZpc{}}
    \PY{n+nf}{group\PYZus{}by}\PY{p}{(}\PY{n}{Piosenka}\PY{p}{)}\PY{o}{\PYZpc{}\PYZgt{}\PYZpc{}}
    \PY{n+nf}{summarize}\PY{p}{(}
        \PY{n}{sumaPunktów} \PY{o}{=} \PY{n+nf}{sum}\PY{p}{(}\PY{n}{Punkty}\PY{p}{)}\PY{p}{,}
        \PY{n}{liczbaWystapien} \PY{o}{=} \PY{n+nf}{n}\PY{p}{(}\PY{p}{)}
    \PY{p}{)}\PY{o}{\PYZpc{}\PYZgt{}\PYZpc{}}
    \PY{n+nf}{summarize}\PY{p}{(}
        ś\PY{n}{r.tyg.na.liscie} \PY{o}{=} \PY{n+nf}{mean}\PY{p}{(}\PY{n}{liczbaWystapien}\PY{p}{)}\PY{p}{,}
        \PY{n}{med.tyg.na.liscie} \PY{o}{=} \PY{n+nf}{median}\PY{p}{(}\PY{n}{liczbaWystapien}\PY{p}{)}\PY{p}{,}
        ś\PY{n}{r.suma.punktów} \PY{o}{=} \PY{n+nf}{mean}\PY{p}{(}\PY{n}{sumaPunktów}\PY{p}{)}\PY{p}{,}
        \PY{n}{med.sumy.punktow} \PY{o}{=} \PY{n+nf}{median}\PY{p}{(}\PY{n}{sumaPunktów}\PY{p}{)}
    \PY{p}{)}
\end{Verbatim}
\end{tcolorbox}

    \begin{tabular}{r|llll}
 śr.tyg.na.liscie & med.tyg.na.liscie & śr.suma.punktów & med.sumy.punktow\\
\hline
	 9.448193 & 8        & 146.3997 & 86      \\
\end{tabular}


    
    Nawet tak prosty test jak wyżej przyniósł ciekawe wyniki! Średnia jest
oddalona od mediany, co w statystyce oznacza, że zbiór jest skrzywiony w
stronę mniejszych wartości.Narysujmy zatem oba histogramy - wykresy
liczby utworów w zależności od liczby tygodni w zestawieniu oraz sumy
zdobytych punktów

    \begin{tcolorbox}[breakable, size=fbox, boxrule=1pt, pad at break*=1mm,colback=cellbackground, colframe=cellborder]
\prompt{In}{incolor}{167}{\boxspacing}
\begin{Verbatim}[commandchars=\\\{\}]
\PY{n}{records} \PY{o}{\PYZpc{}\PYZgt{}\PYZpc{}}
    \PY{n+nf}{transmute}\PY{p}{(}
        \PY{n}{Piosenka} \PY{o}{=} \PY{n+nf}{paste}\PY{p}{(}\PY{n}{Tytuł}\PY{p}{,} \PY{n}{Artysta}\PY{p}{,} \PY{n}{sep}\PY{o}{=}\PY{l+s}{\PYZdq{}}\PY{l+s}{ \PYZhy{} \PYZdq{}}\PY{p}{)}\PY{p}{,}
        \PY{n}{Punkty}\PY{o}{=}\PY{n}{Punkty}
    \PY{p}{)} \PY{o}{\PYZpc{}\PYZgt{}\PYZpc{}}
    \PY{n+nf}{group\PYZus{}by}\PY{p}{(}\PY{n}{Piosenka}\PY{p}{)}\PY{o}{\PYZpc{}\PYZgt{}\PYZpc{}}
    \PY{n+nf}{summarize}\PY{p}{(}
        \PY{n}{liczbaWystapien} \PY{o}{=} \PY{n+nf}{n}\PY{p}{(}\PY{p}{)}\PY{p}{,}
        \PY{p}{)}\PY{o}{\PYZpc{}\PYZgt{}\PYZpc{}}
    \PY{n+nf}{ggplot}\PY{p}{(}\PY{p}{)} \PY{o}{+} \PY{n+nf}{theme\PYZus{}bw}\PY{p}{(}\PY{p}{)}\PY{o}{+}
    \PY{n+nf}{theme}\PY{p}{(}\PY{n}{plot.title} \PY{o}{=} \PY{n+nf}{element\PYZus{}text}\PY{p}{(}\PY{n}{hjust} \PY{o}{=} \PY{l+m}{0.5}\PY{p}{)}\PY{p}{)} \PY{o}{+}
    \PY{n+nf}{geom\PYZus{}bar}\PY{p}{(}\PY{n+nf}{aes}\PY{p}{(}\PY{n}{x}\PY{o}{=}\PY{n}{liczbaWystapien}\PY{p}{)}\PY{p}{,} \PY{n}{stat}\PY{o}{=}\PY{l+s}{\PYZdq{}}\PY{l+s}{bin\PYZdq{}}\PY{p}{,} \PY{n}{binwidth}\PY{o}{=}\PY{l+m}{1}\PY{p}{,} \PY{n}{fill}\PY{o}{=}\PY{l+s}{\PYZdq{}}\PY{l+s}{grey\PYZdq{}}\PY{p}{)} \PY{o}{+}
    \PY{n+nf}{ylab}\PY{p}{(}\PY{l+s}{\PYZdq{}}\PY{l+s}{Liczba utworów\PYZdq{}}\PY{p}{)} \PY{o}{+}
    \PY{n+nf}{ggtitle}\PY{p}{(}\PY{l+s}{\PYZdq{}}\PY{l+s}{Liczba utworów w zależności od liczby wystąpień na liście\PYZdq{}}\PY{p}{)}
\end{Verbatim}
\end{tcolorbox}

    \begin{center}
    \adjustimage{max size={0.9\linewidth}{0.9\paperheight}}{Analiza_Listy_Przebojow_Trojki_files/Analiza_Listy_Przebojow_Trojki_16_0.png}
    \end{center}
    { \hspace*{\fill} \\}
    
    \begin{tcolorbox}[breakable, size=fbox, boxrule=1pt, pad at break*=1mm,colback=cellbackground, colframe=cellborder]
\prompt{In}{incolor}{168}{\boxspacing}
\begin{Verbatim}[commandchars=\\\{\}]
\PY{n}{records} \PY{o}{\PYZpc{}\PYZgt{}\PYZpc{}}
    \PY{n+nf}{transmute}\PY{p}{(}
        \PY{n}{Piosenka} \PY{o}{=} \PY{n+nf}{paste}\PY{p}{(}\PY{n}{Tytuł}\PY{p}{,} \PY{n}{Artysta}\PY{p}{,} \PY{n}{sep}\PY{o}{=}\PY{l+s}{\PYZdq{}}\PY{l+s}{ \PYZhy{} \PYZdq{}}\PY{p}{)}\PY{p}{,}
        \PY{n}{Punkty}\PY{o}{=}\PY{n}{Punkty}
    \PY{p}{)} \PY{o}{\PYZpc{}\PYZgt{}\PYZpc{}}
    \PY{n+nf}{group\PYZus{}by}\PY{p}{(}\PY{n}{Piosenka}\PY{p}{)}\PY{o}{\PYZpc{}\PYZgt{}\PYZpc{}}
    \PY{n+nf}{summarize}\PY{p}{(}
        \PY{n}{sumaPunktów} \PY{o}{=} \PY{n+nf}{sum}\PY{p}{(}\PY{n}{Punkty}\PY{p}{)}\PY{p}{,}
        \PY{p}{)}\PY{o}{\PYZpc{}\PYZgt{}\PYZpc{}}
    \PY{n+nf}{ggplot}\PY{p}{(}\PY{p}{)} \PY{o}{+}
    \PY{n+nf}{theme\PYZus{}bw}\PY{p}{(}\PY{p}{)}\PY{o}{+}
    \PY{n+nf}{theme}\PY{p}{(}\PY{n}{plot.title} \PY{o}{=} \PY{n+nf}{element\PYZus{}text}\PY{p}{(}\PY{n}{hjust} \PY{o}{=} \PY{l+m}{0.5}\PY{p}{)}\PY{p}{)} \PY{o}{+}
    \PY{n+nf}{geom\PYZus{}bar}\PY{p}{(}\PY{n+nf}{aes}\PY{p}{(}\PY{n}{x}\PY{o}{=}\PY{n}{sumaPunktów}\PY{p}{)}\PY{p}{,} \PY{n}{stat}\PY{o}{=}\PY{l+s}{\PYZdq{}}\PY{l+s}{bin\PYZdq{}}\PY{p}{,} \PY{n}{binwidth}\PY{o}{=}\PY{l+m}{10}\PY{p}{,} \PY{n}{fill}\PY{o}{=}\PY{l+s}{\PYZsq{}}\PY{l+s}{darkgreen\PYZsq{}}\PY{p}{)} \PY{o}{+} \PY{n+nf}{ylab}\PY{p}{(}\PY{l+s}{\PYZdq{}}\PY{l+s}{Liczba utworów\PYZdq{}}\PY{p}{)} \PY{o}{+}
    \PY{n+nf}{ggtitle}\PY{p}{(}\PY{l+s}{\PYZdq{}}\PY{l+s}{Liczba utworów w zależności od sumy zebranych punktów\PYZdq{}}\PY{p}{)}
\end{Verbatim}
\end{tcolorbox}

    \begin{center}
    \adjustimage{max size={0.9\linewidth}{0.9\paperheight}}{Analiza_Listy_Przebojow_Trojki_files/Analiza_Listy_Przebojow_Trojki_17_0.png}
    \end{center}
    { \hspace*{\fill} \\}
    
    Teraz od razu widać prostą zależność - zdecydowana większość utworów
znajduje się w zestawieniu zajmuje niskie pozycje, uzyskując w ten
sposób niską sumaryczną punktację. Niewielka liczba utworów przełamuje
barierę 500 punktów, jeszcze mniej 1000. Sprawdźmy zatem, jak wygląda ów
``top'' utworów.

\hypertarget{top-utworuxf3w-i-artystuxf3w-listy-przebojuxf3w-truxf3jki}{%
\subsubsection{TOP utworów i artystów Listy Przebojów
Trójki}\label{top-utworuxf3w-i-artystuxf3w-listy-przebojuxf3w-truxf3jki}}

Poniżej zestawiono 20 utworów o największej sumarycznej liczbie punktów

    \begin{tcolorbox}[breakable, size=fbox, boxrule=1pt, pad at break*=1mm,colback=cellbackground, colframe=cellborder]
\prompt{In}{incolor}{10}{\boxspacing}
\begin{Verbatim}[commandchars=\\\{\}]
\PY{n}{records} \PY{o}{\PYZpc{}\PYZgt{}\PYZpc{}}
    \PY{n+nf}{transmute}\PY{p}{(}
        \PY{n}{Piosenka} \PY{o}{=} \PY{n+nf}{paste}\PY{p}{(}\PY{n}{Tytuł}\PY{p}{,} \PY{n}{Artysta}\PY{p}{,} \PY{n}{sep}\PY{o}{=}\PY{l+s}{\PYZdq{}}\PY{l+s}{ \PYZhy{} \PYZdq{}}\PY{p}{)}\PY{p}{,}
        \PY{n}{Punkty}\PY{o}{=}\PY{n}{Punkty}
    \PY{p}{)}\PY{o}{\PYZpc{}\PYZgt{}\PYZpc{}}
    \PY{n+nf}{group\PYZus{}by}\PY{p}{(}\PY{n}{Piosenka}\PY{p}{)} \PY{o}{\PYZpc{}\PYZgt{}\PYZpc{}}
    \PY{n+nf}{summarise}\PY{p}{(}
          \PY{n}{SumaPunktów} \PY{o}{=} \PY{n+nf}{sum}\PY{p}{(}\PY{n}{Punkty}\PY{p}{)}
          \PY{p}{)} \PY{o}{\PYZpc{}\PYZgt{}\PYZpc{}}
    \PY{n+nf}{top\PYZus{}n}\PY{p}{(}\PY{l+m}{20}\PY{p}{,} \PY{n}{SumaPunktów}\PY{p}{)} \PY{o}{\PYZpc{}\PYZgt{}\PYZpc{}}
    
    \PY{n+nf}{ggplot}\PY{p}{(}\PY{p}{)} \PY{o}{+}
    \PY{n+nf}{geom\PYZus{}bar}\PY{p}{(}\PY{n+nf}{aes}\PY{p}{(}\PY{n}{x} \PY{o}{=} \PY{n+nf}{reorder}\PY{p}{(}\PY{n}{Piosenka}\PY{p}{,} \PY{o}{\PYZhy{}}\PY{n}{SumaPunktów}\PY{p}{)}\PY{p}{,} \PY{n}{y} \PY{o}{=} \PY{n}{SumaPunktów}\PY{p}{)}\PY{p}{,} \PY{n}{stat}\PY{o}{=}\PY{l+s}{\PYZdq{}}\PY{l+s}{identity\PYZdq{}}\PY{p}{,} \PY{n}{fill}\PY{o}{=}\PY{l+s}{\PYZsq{}}\PY{l+s}{blue\PYZsq{}}\PY{p}{)}\PY{o}{+}
    \PY{n+nf}{theme\PYZus{}bw}\PY{p}{(}\PY{p}{)}\PY{o}{+}
    \PY{n+nf}{theme}\PY{p}{(}\PY{n}{axis.text.x} \PY{o}{=} \PY{n+nf}{element\PYZus{}text}\PY{p}{(}\PY{n}{angle}\PY{o}{=}\PY{l+m}{90}\PY{p}{,} \PY{n}{hjust}\PY{o}{=}\PY{l+m}{1}\PY{p}{)}\PY{p}{,} \PY{n}{plot.title} \PY{o}{=} \PY{n+nf}{element\PYZus{}text}\PY{p}{(}\PY{n}{hjust} \PY{o}{=} \PY{l+m}{0.5}\PY{p}{)}\PY{p}{)} \PY{o}{+}
    \PY{n+nf}{xlab}\PY{p}{(}\PY{l+s}{\PYZdq{}}\PY{l+s}{Piosenka \PYZhy{} Wykonawca\PYZdq{}}\PY{p}{)}\PY{o}{+}
    \PY{n+nf}{ggtitle}\PY{p}{(}\PY{l+s}{\PYZdq{}}\PY{l+s}{Top 20 utworów o największej liczbie zebranych punktów\PYZdq{}}\PY{p}{)}
\end{Verbatim}
\end{tcolorbox}

    \begin{center}
    \adjustimage{max size={0.9\linewidth}{0.9\paperheight}}{Analiza_Listy_Przebojow_Trojki_files/Analiza_Listy_Przebojow_Trojki_19_0.png}
    \end{center}
    { \hspace*{\fill} \\}
    
    Sprawdźmy również, ile dokładnie utworów przekroczyło bariery okrągłej
liczby punktów

    \begin{tcolorbox}[breakable, size=fbox, boxrule=1pt, pad at break*=1mm,colback=cellbackground, colframe=cellborder]
\prompt{In}{incolor}{151}{\boxspacing}
\begin{Verbatim}[commandchars=\\\{\}]
\PY{n}{tableWithPoints} \PY{o}{\PYZlt{}\PYZhy{}}    
    \PY{n}{records} \PY{o}{\PYZpc{}\PYZgt{}\PYZpc{}} 
    \PY{n+nf}{transmute}\PY{p}{(}
        \PY{n}{Piosenka} \PY{o}{=} \PY{n+nf}{paste}\PY{p}{(}\PY{n}{Tytuł}\PY{p}{,} \PY{n}{Artysta}\PY{p}{,} \PY{n}{sep}\PY{o}{=}\PY{l+s}{\PYZdq{}}\PY{l+s}{ \PYZhy{} \PYZdq{}}\PY{p}{)}\PY{p}{,}
        \PY{n}{Punkty}\PY{o}{=}\PY{n}{Punkty}
    \PY{p}{)}\PY{o}{\PYZpc{}\PYZgt{}\PYZpc{}}
    \PY{n+nf}{group\PYZus{}by}\PY{p}{(}\PY{n}{Piosenka}\PY{p}{)} \PY{o}{\PYZpc{}\PYZgt{}\PYZpc{}}
    \PY{n+nf}{summarise}\PY{p}{(}\PY{n}{SumaPunktów} \PY{o}{=} \PY{n+nf}{sum}\PY{p}{(}\PY{n}{Punkty}\PY{p}{)}\PY{p}{)}

\PY{n}{numberOfSongs} \PY{o}{\PYZlt{}\PYZhy{}}
    \PY{n}{records} \PY{o}{\PYZpc{}\PYZgt{}\PYZpc{}}
    \PY{n+nf}{transmute}\PY{p}{(}\PY{n}{Piosenka} \PY{o}{=} \PY{n+nf}{paste}\PY{p}{(}\PY{n}{Tytuł}\PY{p}{,} \PY{n}{Artysta}\PY{p}{,} \PY{n}{sep}\PY{o}{=}\PY{l+s}{\PYZdq{}}\PY{l+s}{ \PYZhy{} \PYZdq{}}\PY{p}{)}\PY{p}{)}\PY{o}{\PYZpc{}\PYZgt{}\PYZpc{}}
    \PY{n+nf}{distinct}\PY{p}{(}\PY{p}{)} \PY{o}{\PYZpc{}\PYZgt{}\PYZpc{}}
    \PY{n+nf}{summarise}\PY{p}{(}\PY{n}{liczbaPiosenek} \PY{o}{=} \PY{n+nf}{n}\PY{p}{(}\PY{p}{)}\PY{p}{)}


\PY{n}{progi} \PY{o}{=} \PY{n+nf}{seq}\PY{p}{(}\PY{l+m}{0}\PY{p}{,} \PY{l+m}{1000}\PY{p}{,} \PY{n}{by}\PY{o}{=}\PY{l+m}{100}\PY{p}{)}
\PY{n}{liczbaPiosenekPowyzejProgu} \PY{o}{=} \PY{n+nf}{c}\PY{p}{(}\PY{p}{)}
\PY{n}{procent} \PY{o}{=} \PY{n+nf}{c}\PY{p}{(}\PY{p}{)}
\PY{n}{res} \PY{o}{\PYZlt{}\PYZhy{}} \PY{n+nf}{data.frame}\PY{p}{(}\PY{p}{)}
\PY{n+nf}{for }\PY{p}{(}\PY{n}{Próg} \PY{n}{in} \PY{n}{progi}\PY{p}{)}\PY{p}{\PYZob{}}
    
    \PY{n}{tmp} \PY{o}{\PYZlt{}\PYZhy{}}
        \PY{n}{tableWithPoints} \PY{o}{\PYZpc{}\PYZgt{}\PYZpc{}}
        \PY{n+nf}{filter}\PY{p}{(}\PY{n}{SumaPunktów} \PY{o}{\PYZgt{}} \PY{n}{Próg}\PY{p}{)} \PY{o}{\PYZpc{}\PYZgt{}\PYZpc{}}
        \PY{n+nf}{summarize}\PY{p}{(}\PY{n}{Liczba.Piosenek} \PY{o}{=} \PY{n+nf}{n}\PY{p}{(}\PY{p}{)}\PY{p}{)}
    \PY{n}{pr} \PY{o}{\PYZlt{}\PYZhy{}} \PY{n+nf}{round}\PY{p}{(}\PY{n}{tmp}\PY{o}{*}\PY{l+m}{100}\PY{o}{/}\PY{n}{numberOfSongs}\PY{p}{,} \PY{l+m}{2}\PY{p}{)}
    \PY{n}{newRow} \PY{o}{\PYZlt{}\PYZhy{}} \PY{n+nf}{cbind}\PY{p}{(}\PY{n}{Próg}\PY{p}{,} \PY{n}{tmp}\PY{p}{,} \PY{n}{pr}\PY{p}{)}
    \PY{n}{res} \PY{o}{\PYZlt{}\PYZhy{}} \PY{n+nf}{rbind}\PY{p}{(}\PY{n}{res}\PY{p}{,} \PY{n}{newRow}\PY{p}{)}
\PY{p}{\PYZcb{}}
\PY{n+nf}{names}\PY{p}{(}\PY{n}{res}\PY{p}{)}\PY{n}{[3}\PY{n}{]} \PY{o}{\PYZlt{}\PYZhy{}} \PY{l+s}{\PYZdq{}}\PY{l+s}{Procent\PYZdq{}}
\PY{n}{res}
\end{Verbatim}
\end{tcolorbox}

    \begin{tabular}{r|lll}
 Próg & Liczba.Piosenek & Procent\\
\hline
	    0   & 6225   & 100.00\\
	  100   & 2844   &  45.69\\
	  200   & 1677   &  26.94\\
	  300   &  972   &  15.61\\
	  400   &  548   &   8.80\\
	  500   &  291   &   4.67\\
	  600   &  159   &   2.55\\
	  700   &   79   &   1.27\\
	  800   &   38   &   0.61\\
	  900   &   19   &   0.31\\
	 1000   &    7   &   0.11\\
\end{tabular}


    
    Podobny jak wyżej histogram można również stworzyć dla liczby punktów
zdobytych przez danego artystę

    \begin{tcolorbox}[breakable, size=fbox, boxrule=1pt, pad at break*=1mm,colback=cellbackground, colframe=cellborder]
\prompt{In}{incolor}{152}{\boxspacing}
\begin{Verbatim}[commandchars=\\\{\}]
\PY{n}{records} \PY{o}{\PYZpc{}\PYZgt{}\PYZpc{}}
    \PY{n+nf}{select}\PY{p}{(}\PY{n}{Artysta}\PY{p}{,} \PY{n}{Punkty}\PY{p}{)} \PY{o}{\PYZpc{}\PYZgt{}\PYZpc{}}
    \PY{n+nf}{group\PYZus{}by}\PY{p}{(}\PY{n}{Artysta}\PY{p}{)} \PY{o}{\PYZpc{}\PYZgt{}\PYZpc{}}
    \PY{n+nf}{summarise}\PY{p}{(}
        \PY{n}{SumaPunktów} \PY{o}{=} \PY{n+nf}{sum}\PY{p}{(}\PY{n}{Punkty}\PY{p}{)}
        \PY{p}{)}\PY{o}{\PYZpc{}\PYZgt{}\PYZpc{}}
    \PY{n+nf}{top\PYZus{}n}\PY{p}{(}\PY{l+m}{20}\PY{p}{,} \PY{n}{SumaPunktów}\PY{p}{)} \PY{o}{\PYZpc{}\PYZgt{}\PYZpc{}}
    
    \PY{n+nf}{ggplot}\PY{p}{(}\PY{p}{)} \PY{o}{+}
    \PY{n+nf}{geom\PYZus{}bar}\PY{p}{(}\PY{n+nf}{aes}\PY{p}{(}\PY{n}{x} \PY{o}{=} \PY{n+nf}{reorder}\PY{p}{(}\PY{n}{Artysta}\PY{p}{,} \PY{o}{\PYZhy{}}\PY{n}{SumaPunktów}\PY{p}{)}\PY{p}{,} \PY{n}{y} \PY{o}{=} \PY{n}{SumaPunktów}\PY{p}{)}\PY{p}{,} \PY{n}{stat}\PY{o}{=}\PY{l+s}{\PYZdq{}}\PY{l+s}{identity\PYZdq{}}\PY{p}{,} \PY{n}{fill}\PY{o}{=}\PY{l+s}{\PYZsq{}}\PY{l+s}{lightblue\PYZsq{}}\PY{p}{)}\PY{o}{+}
    \PY{n+nf}{theme\PYZus{}bw}\PY{p}{(}\PY{p}{)}\PY{o}{+}
    \PY{n+nf}{theme}\PY{p}{(}\PY{n}{axis.text.x} \PY{o}{=} \PY{n+nf}{element\PYZus{}text}\PY{p}{(}\PY{n}{angle}\PY{o}{=}\PY{l+m}{45}\PY{p}{,} \PY{n}{hjust}\PY{o}{=}\PY{l+m}{1}\PY{p}{)}\PY{p}{,} \PY{n}{plot.title} \PY{o}{=} \PY{n+nf}{element\PYZus{}text}\PY{p}{(}\PY{n}{hjust} \PY{o}{=} \PY{l+m}{0.5}\PY{p}{)}\PY{p}{)} \PY{o}{+}
    \PY{n+nf}{xlab}\PY{p}{(}\PY{l+s}{\PYZdq{}}\PY{l+s}{Wykonawca\PYZdq{}}\PY{p}{)}\PY{o}{+}
    \PY{n+nf}{ggtitle}\PY{p}{(}\PY{l+s}{\PYZdq{}}\PY{l+s}{Top 20 artystów o największej liczbie zebranych punktów\PYZdq{}}\PY{p}{)}
\end{Verbatim}
\end{tcolorbox}

    \begin{center}
    \adjustimage{max size={0.9\linewidth}{0.9\paperheight}}{Analiza_Listy_Przebojow_Trojki_files/Analiza_Listy_Przebojow_Trojki_23_0.png}
    \end{center}
    { \hspace*{\fill} \\}
    
    Jak suma punktów rozkładała się dla podanych wyżej twórców w ciągu
kolejnych notowań?

    \begin{tcolorbox}[breakable, size=fbox, boxrule=1pt, pad at break*=1mm,colback=cellbackground, colframe=cellborder]
\prompt{In}{incolor}{153}{\boxspacing}
\begin{Verbatim}[commandchars=\\\{\}]
\PY{n}{topArtystow} \PY{o}{\PYZlt{}\PYZhy{}}
    \PY{n}{records} \PY{o}{\PYZpc{}\PYZgt{}\PYZpc{}}
    \PY{n+nf}{select}\PY{p}{(}\PY{n}{Artysta}\PY{p}{,} \PY{n}{Punkty}\PY{p}{)} \PY{o}{\PYZpc{}\PYZgt{}\PYZpc{}}
    \PY{n+nf}{group\PYZus{}by}\PY{p}{(}\PY{n}{Artysta}\PY{p}{)} \PY{o}{\PYZpc{}\PYZgt{}\PYZpc{}}
    \PY{n+nf}{summarise}\PY{p}{(}
        \PY{n}{SumaPunktów} \PY{o}{=} \PY{n+nf}{sum}\PY{p}{(}\PY{n}{Punkty}\PY{p}{)}
        \PY{p}{)}\PY{o}{\PYZpc{}\PYZgt{}\PYZpc{}}
    \PY{n+nf}{top\PYZus{}n}\PY{p}{(}\PY{l+m}{10}\PY{p}{,} \PY{n}{SumaPunktów}\PY{p}{)}

\PY{n}{najlepsiArtysci} \PY{o}{\PYZlt{}\PYZhy{}} \PY{n}{topArtystow}\PY{o}{\PYZdl{}}\PY{n}{Artysta}

\PY{n}{records} \PY{o}{\PYZpc{}\PYZgt{}\PYZpc{}}
\PY{n+nf}{filter}\PY{p}{(}\PY{n}{Artysta} \PY{o}{\PYZpc{}in\PYZpc{}} \PY{n}{najlepsiArtysci}\PY{p}{)} \PY{o}{\PYZpc{}\PYZgt{}\PYZpc{}}
\PY{n+nf}{group\PYZus{}by}\PY{p}{(}\PY{n}{Artysta}\PY{p}{)} \PY{o}{\PYZpc{}\PYZgt{}\PYZpc{}}
\PY{n+nf}{mutate}\PY{p}{(}\PY{n}{KumulowanePunkty}\PY{o}{=}\PY{n+nf}{cumsum}\PY{p}{(}\PY{n}{Punkty}\PY{p}{)}\PY{p}{)} \PY{o}{\PYZpc{}\PYZgt{}\PYZpc{}}
\PY{n+nf}{select}\PY{p}{(}\PY{n}{Nr.notowania}\PY{p}{,} \PY{n}{Artysta}\PY{p}{,} \PY{n}{KumulowanePunkty}\PY{p}{)} \PY{o}{\PYZpc{}\PYZgt{}\PYZpc{}}
\PY{n+nf}{ggplot}\PY{p}{(}\PY{p}{)} \PY{o}{+}
\PY{n+nf}{theme\PYZus{}bw}\PY{p}{(}\PY{p}{)} \PY{o}{+}
\PY{n+nf}{theme}\PY{p}{(}\PY{n}{plot.title} \PY{o}{=} \PY{n+nf}{element\PYZus{}text}\PY{p}{(}\PY{n}{hjust} \PY{o}{=} \PY{l+m}{0.5}\PY{p}{)}\PY{p}{)} \PY{o}{+}
\PY{n+nf}{geom\PYZus{}line}\PY{p}{(}\PY{n+nf}{aes}\PY{p}{(}\PY{n}{Nr.notowania}\PY{p}{,} \PY{n}{KumulowanePunkty}\PY{p}{,} \PY{n}{color}\PY{o}{=}\PY{n}{Artysta}\PY{p}{)}\PY{p}{)} \PY{o}{+}
\PY{n+nf}{ggtitle}\PY{p}{(}\PY{l+s}{\PYZdq{}}\PY{l+s}{Kumulacyjny wykres najlepszych zespołów\PYZdq{}}\PY{p}{)}
\end{Verbatim}
\end{tcolorbox}

    \begin{center}
    \adjustimage{max size={0.9\linewidth}{0.9\paperheight}}{Analiza_Listy_Przebojow_Trojki_files/Analiza_Listy_Przebojow_Trojki_25_0.png}
    \end{center}
    { \hspace*{\fill} \\}
    
    Powyższy wykres pokazuje nam o wiele dokładniejsze dane niż sam
histogram. Teraz widać wyraźnie, że artyści tacy jak Maanam czy Lady
Pank ``zapewnili'' sobie najwyższe miejsca już w pierwszych notowaniach,
natomiast nowsze zespoły, takie jak np. Lao Che wspinały się dynamicznie
w ciągu ostatnich kilkuset notowań.

\hypertarget{najlepsze-utwory-poszczeguxf3lnych-wykonawcuxf3w}{%
\subsubsection{Najlepsze utwory poszczególnych
wykonawców}\label{najlepsze-utwory-poszczeguxf3lnych-wykonawcuxf3w}}

Również ``skoki'' wskazują na dużą liczbę punktów zdobywanych w którkim
czasie, tzw. ``hitów''. Jakie piosenki przyniosły njwięcej punktów dla
poszczgólnych artystów?

    \begin{tcolorbox}[breakable, size=fbox, boxrule=1pt, pad at break*=1mm,colback=cellbackground, colframe=cellborder]
\prompt{In}{incolor}{195}{\boxspacing}
\begin{Verbatim}[commandchars=\\\{\}]
\PY{n}{topArtystow} \PY{o}{\PYZlt{}\PYZhy{}}
    \PY{n}{records} \PY{o}{\PYZpc{}\PYZgt{}\PYZpc{}}
    \PY{n+nf}{select}\PY{p}{(}\PY{n}{Artysta}\PY{p}{,} \PY{n}{Punkty}\PY{p}{)} \PY{o}{\PYZpc{}\PYZgt{}\PYZpc{}}
    \PY{n+nf}{group\PYZus{}by}\PY{p}{(}\PY{n}{Artysta}\PY{p}{)} \PY{o}{\PYZpc{}\PYZgt{}\PYZpc{}}
    \PY{n+nf}{summarise}\PY{p}{(}
        \PY{n}{SumaPunktów} \PY{o}{=} \PY{n+nf}{sum}\PY{p}{(}\PY{n}{Punkty}\PY{p}{)}
        \PY{p}{)}\PY{o}{\PYZpc{}\PYZgt{}\PYZpc{}}
    \PY{n+nf}{top\PYZus{}n}\PY{p}{(}\PY{l+m}{10}\PY{p}{,} \PY{n}{SumaPunktów}\PY{p}{)}

\PY{n}{najlepsiArtysci} \PY{o}{\PYZlt{}\PYZhy{}} \PY{n}{topArtystow}\PY{o}{\PYZdl{}}\PY{n}{Artysta}

\PY{n}{records} \PY{o}{\PYZpc{}\PYZgt{}\PYZpc{}}
\PY{n+nf}{filter}\PY{p}{(}\PY{n}{Artysta} \PY{o}{\PYZpc{}in\PYZpc{}} \PY{n}{najlepsiArtysci}\PY{p}{)} \PY{o}{\PYZpc{}\PYZgt{}\PYZpc{}}
\PY{n+nf}{group\PYZus{}by}\PY{p}{(}\PY{n}{Artysta}\PY{p}{,} \PY{n}{Tytuł}\PY{p}{)} \PY{o}{\PYZpc{}\PYZgt{}\PYZpc{}}
\PY{n+nf}{summarise}\PY{p}{(}
    \PY{n}{SumaPunktów} \PY{o}{=} \PY{n+nf}{sum}\PY{p}{(}\PY{n}{Punkty}\PY{p}{)}
    \PY{p}{)} \PY{o}{\PYZpc{}\PYZgt{}\PYZpc{}}
\PY{n+nf}{ungroup}\PY{p}{(}\PY{p}{)} \PY{o}{\PYZpc{}\PYZgt{}\PYZpc{}}
\PY{n+nf}{group\PYZus{}by}\PY{p}{(}\PY{n}{Artysta}\PY{p}{)} \PY{o}{\PYZpc{}\PYZgt{}\PYZpc{}}
\PY{n+nf}{arrange}\PY{p}{(}\PY{n+nf}{desc}\PY{p}{(}\PY{n}{SumaPunktów}\PY{p}{)}\PY{p}{)} \PY{o}{\PYZpc{}\PYZgt{}\PYZpc{}}
\PY{n+nf}{top\PYZus{}n}\PY{p}{(}\PY{l+m}{5}\PY{p}{,} \PY{n}{SumaPunktów}\PY{p}{)} \PY{o}{\PYZpc{}\PYZgt{}\PYZpc{}}
\PY{c+c1}{\PYZsh{}arrange(Artysta)\PYZpc{}\PYZgt{}\PYZpc{}}
\PY{n+nf}{ggplot}\PY{p}{(}\PY{p}{)} \PY{o}{+}
\PY{n+nf}{geom\PYZus{}bar}\PY{p}{(}\PY{n+nf}{aes}\PY{p}{(}\PY{n}{x} \PY{o}{=} \PY{n+nf}{reorder}\PY{p}{(}\PY{n}{Tytuł}\PY{p}{,} \PY{n}{SumaPunktów}\PY{p}{)}\PY{p}{,} \PY{n}{SumaPunktów}\PY{p}{,} \PY{n}{color}\PY{o}{=}\PY{n}{Artysta}\PY{p}{)}\PY{p}{,} \PY{n}{stat}\PY{o}{=}\PY{l+s}{\PYZdq{}}\PY{l+s}{identity\PYZdq{}}\PY{p}{,} \PY{n}{size}\PY{o}{=}\PY{l+m}{0.9}\PY{p}{)}\PY{o}{+}
\PY{n+nf}{theme}\PY{p}{(}\PY{n}{axis.ticks.length.y}\PY{o}{=}\PY{n+nf}{unit}\PY{p}{(}\PY{l+m}{0.25}\PY{p}{,} \PY{l+s}{\PYZdq{}}\PY{l+s}{cm\PYZdq{}}\PY{p}{)}\PY{p}{,} \PY{n}{axis.text.y} \PY{o}{=} \PY{n+nf}{element\PYZus{}text}\PY{p}{(}\PY{n}{size}\PY{o}{=}\PY{l+m}{6}\PY{p}{,} \PY{n}{hjust}\PY{o}{=}\PY{l+m}{1}\PY{p}{)}\PY{p}{)}\PY{o}{+}
\PY{n+nf}{facet\PYZus{}wrap}\PY{p}{(}\PY{o}{\PYZti{}}\PY{n}{Artysta}\PY{p}{,} \PY{n}{scales}\PY{o}{=}\PY{l+s}{\PYZdq{}}\PY{l+s}{free\PYZus{}y\PYZdq{}}\PY{p}{,} \PY{n}{ncol}\PY{o}{=}\PY{l+m}{1}\PY{p}{)}\PY{o}{+} \PY{n+nf}{coord\PYZus{}flip}\PY{p}{(}\PY{p}{)} \PY{o}{+}
\PY{n+nf}{xlab}\PY{p}{(}\PY{l+s}{\PYZdq{}}\PY{l+s}{Tytuły piosenek\PYZdq{}}\PY{p}{)} \PY{o}{+}
\PY{n+nf}{ggtitle}\PY{p}{(}\PY{l+s}{\PYZdq{}}\PY{l+s}{Najlepiej punktowane utwory artystów\PYZdq{}}\PY{p}{)}
\end{Verbatim}
\end{tcolorbox}

    \begin{center}
    \adjustimage{max size={0.9\linewidth}{0.9\paperheight}}{Analiza_Listy_Przebojow_Trojki_files/Analiza_Listy_Przebojow_Trojki_27_0.png}
    \end{center}
    { \hspace*{\fill} \\}
    
    \hypertarget{badanie-hipotez}{%
\subsection{Badanie hipotez}\label{badanie-hipotez}}

\hypertarget{mnie-siux119-podobajux105-melodie-ktuxf3re-juux17c-raz-sux142yszaux142em-czyli---czy-znane-nazwiska-artystuxf3w-zapewniajux105-wyux17csze-pozycje}{%
\subsubsection{\texorpdfstring{\emph{Mnie się podobają melodie, które
już raz słyszałem}, czyli - Czy znane nazwiska artystów zapewniają
wyższe
pozycje?}{Mnie się podobają melodie, które już raz słyszałem, czyli - Czy znane nazwiska artystów zapewniają wyższe pozycje?}}\label{mnie-siux119-podobajux105-melodie-ktuxf3re-juux17c-raz-sux142yszaux142em-czyli---czy-znane-nazwiska-artystuxf3w-zapewniajux105-wyux17csze-pozycje}}

\hypertarget{gux142uxf3wna-myux15bl}{%
\paragraph{Główna myśl}\label{gux142uxf3wna-myux15bl}}

Aby znaleźć (przybliżoną) odpowiedź na powyższe pytanie, można
sprawdzić, czy istnieje korelacja między liczbą piosenek w zestawieniu
danego artysty, a liczbą punktów na dany utwór

    \begin{tcolorbox}[breakable, size=fbox, boxrule=1pt, pad at break*=1mm,colback=cellbackground, colframe=cellborder]
\prompt{In}{incolor}{194}{\boxspacing}
\begin{Verbatim}[commandchars=\\\{\}]
\PY{c+c1}{\PYZsh{}Artyści wg liczby piosenek w liście}
\PY{n}{liczbaPiosenekNaArtyste} \PY{o}{\PYZlt{}\PYZhy{}} 
    \PY{n}{records} \PY{o}{\PYZpc{}\PYZgt{}\PYZpc{}}
    \PY{n+nf}{select}\PY{p}{(}\PY{n}{Tytuł}\PY{p}{,} \PY{n}{Artysta}\PY{p}{)} \PY{o}{\PYZpc{}\PYZgt{}\PYZpc{}}
    \PY{n+nf}{distinct}\PY{p}{(}\PY{p}{)} \PY{o}{\PYZpc{}\PYZgt{}\PYZpc{}}
    \PY{n+nf}{group\PYZus{}by}\PY{p}{(}\PY{n}{Artysta}\PY{p}{)} \PY{o}{\PYZpc{}\PYZgt{}\PYZpc{}}
    \PY{n+nf}{summarise}\PY{p}{(}\PY{n}{Liczba.utworów.na.liście} \PY{o}{=} \PY{n+nf}{n}\PY{p}{(}\PY{p}{)}\PY{p}{)} \PY{o}{\PYZpc{}\PYZgt{}\PYZpc{}}
    \PY{n+nf}{arrange}\PY{p}{(}\PY{n+nf}{desc}\PY{p}{(}\PY{n}{Liczba.utworów.na.liście}\PY{p}{)}\PY{p}{)} \PY{o}{\PYZpc{}\PYZgt{}\PYZpc{}}
    \PY{n+nf}{ungroup}\PY{p}{(}\PY{p}{)}

\PY{c+c1}{\PYZsh{}Średnia punktów na utwór}
ś\PY{n}{rPktNaArtyste} \PY{o}{\PYZlt{}\PYZhy{}} \PY{n}{records} \PY{o}{\PYZpc{}\PYZgt{}\PYZpc{}}
    \PY{n+nf}{select}\PY{p}{(}\PY{n}{Artysta}\PY{p}{,} \PY{n}{Tytuł}\PY{p}{,} \PY{n}{Punkty}\PY{p}{)} \PY{o}{\PYZpc{}\PYZgt{}\PYZpc{}}
    \PY{n+nf}{group\PYZus{}by}\PY{p}{(}\PY{n}{Artysta}\PY{p}{,} \PY{n}{Tytuł}\PY{p}{)} \PY{o}{\PYZpc{}\PYZgt{}\PYZpc{}}
    \PY{n+nf}{summarise}\PY{p}{(}
        \PY{n}{SumaPunktów} \PY{o}{=} \PY{n+nf}{sum}\PY{p}{(}\PY{n}{Punkty}\PY{p}{)}\PY{p}{,}   \PY{c+c1}{\PYZsh{}sumaryczna liczba punktów dla każdego utworu}
        \PY{p}{)} \PY{o}{\PYZpc{}\PYZgt{}\PYZpc{}}
    \PY{n+nf}{ungroup}\PY{p}{(}\PY{p}{)} \PY{o}{\PYZpc{}\PYZgt{}\PYZpc{}}
    \PY{n+nf}{group\PYZus{}by}\PY{p}{(}\PY{n}{Artysta}\PY{p}{)} \PY{o}{\PYZpc{}\PYZgt{}\PYZpc{}}
    \PY{n+nf}{summarise}\PY{p}{(}
        Ś\PY{n}{redniaPunktówNaUtwór} \PY{o}{=} \PY{n+nf}{mean}\PY{p}{(}\PY{n}{SumaPunktów}\PY{p}{)}
    \PY{p}{)} \PY{o}{\PYZpc{}\PYZgt{}\PYZpc{}} \PY{n+nf}{arrange}\PY{p}{(}\PY{n+nf}{desc}\PY{p}{(}Ś\PY{n}{redniaPunktówNaUtwór}\PY{p}{)}\PY{p}{)}\PY{o}{\PYZpc{}\PYZgt{}\PYZpc{}}
    \PY{n+nf}{ungroup}\PY{p}{(}\PY{p}{)}

\PY{n}{joinedTable} \PY{o}{\PYZlt{}\PYZhy{}} \PY{n+nf}{left\PYZus{}join}\PY{p}{(}\PY{n}{liczbaPiosenekNaArtyste}\PY{p}{,} ś\PY{n}{rPktNaArtyste}\PY{p}{,} \PY{n}{by}\PY{o}{=}\PY{l+s}{\PYZdq{}}\PY{l+s}{Artysta\PYZdq{}}\PY{p}{)}
\PY{n}{korelacja} \PY{o}{\PYZlt{}\PYZhy{}} \PY{n+nf}{cor}\PY{p}{(}\PY{n}{joinedTable}\PY{o}{\PYZdl{}}\PY{n}{Liczba.utworów.na.liście}\PY{p}{,} \PY{n}{joinedTable}\PY{o}{\PYZdl{}}Ś\PY{n}{redniaPunktówNaUtwór}\PY{p}{)}

\PY{n}{joinedTable} \PY{o}{\PYZpc{}\PYZgt{}\PYZpc{}}
    \PY{c+c1}{\PYZsh{}top\PYZus{}n(20, Liczba.utworów.na.liście) \PYZpc{}\PYZgt{}\PYZpc{}}
    \PY{n+nf}{ggplot}\PY{p}{(}\PY{n+nf}{aes}\PY{p}{(}\PY{n}{x} \PY{o}{=} \PY{n}{Liczba.utworów.na.liście}\PY{p}{,} \PY{n}{y}\PY{o}{=}Ś\PY{n}{redniaPunktówNaUtwór}\PY{p}{)}\PY{p}{)} \PY{o}{+}
    \PY{n+nf}{geom\PYZus{}point}\PY{p}{(}\PY{p}{)} \PY{o}{+}
    \PY{n+nf}{stat\PYZus{}smooth}\PY{p}{(}\PY{n}{method} \PY{o}{=} \PY{n}{lm}\PY{p}{,} \PY{n}{formula}\PY{o}{=}\PY{n}{y}\PY{o}{\PYZti{}}\PY{n}{x}\PY{p}{)} \PY{o}{+}  \PY{c+c1}{\PYZsh{}poszukiwanie regresji liniowej}
    \PY{n+nf}{theme\PYZus{}bw}\PY{p}{(}\PY{p}{)} \PY{o}{+}
    \PY{n+nf}{ggtitle}\PY{p}{(}\PY{l+s}{\PYZdq{}}\PY{l+s}{Zależność \PYZbs{}\PYZdq{}jakości\PYZbs{}\PYZdq{} utworów wykonawcy od liczby wystąpień w LP3\PYZbs{}nKorelacja:\PYZdq{}} \PY{p}{,} \PY{n}{korelacja}\PY{p}{)}
\end{Verbatim}
\end{tcolorbox}

    \begin{center}
    \adjustimage{max size={0.9\linewidth}{0.9\paperheight}}{Analiza_Listy_Przebojow_Trojki_files/Analiza_Listy_Przebojow_Trojki_29_0.png}
    \end{center}
    { \hspace*{\fill} \\}
    
    Jak widać na powyższym wykresie, korelacja jest znikoma. Linia regresji
prawie w ogóle nie wskazuje punktów znajdujących się na rysunku.

Osobiście uważam, że niski współczynnik korelacji to dobry znak - znani
artyści nie determinują wyglądu listy w stopniu znacznym, dzięki czemu
utalentowani młodzi artyści również mogą znaleźć się na liście.

\hypertarget{spostrzeux17cenie}{%
\paragraph{Spostrzeżenie}\label{spostrzeux17cenie}}

Jednak przyglądając się szczytom powyższych tabel, zauważyłem pewien
szczegół:

    \begin{tcolorbox}[breakable, size=fbox, boxrule=1pt, pad at break*=1mm,colback=cellbackground, colframe=cellborder]
\prompt{In}{incolor}{193}{\boxspacing}
\begin{Verbatim}[commandchars=\\\{\}]
ś\PY{n}{rPktNaArtyste} \PY{o}{\PYZpc{}\PYZgt{}\PYZpc{}} \PY{n+nf}{top\PYZus{}n}\PY{p}{(}\PY{l+m}{20}\PY{p}{,} Ś\PY{n}{redniaPunktówNaUtwór}\PY{p}{)}
\end{Verbatim}
\end{tcolorbox}

    \begin{tabular}{r|ll}
 Artysta & ŚredniaPunktówNaUtwór\\
\hline
	 Gotye feat. Kimbra                                        & 960.00                                                   \\
	 Fisz Emade Tworzywo feat. Justyna Święs                   & 752.00                                                   \\
	 Zbigniew Zamachowski i Grupa MoCarta                      & 746.00                                                   \\
	 Fisz, Emade, Tworzywo feat. Kasia Nosowska                & 682.00                                                   \\
	 Michael Jackson feat. special guitar performance by Slash & 674.00                                                   \\
	 Organek                                                   & 666.75                                                   \\
	 Męskie Granie Orkiestra 2018                              & 660.00                                                   \\
	 Kazik \& Edyta Bartosiewicz                                & 639.00                                                     \\
	 Basia Stępniak-Wilk \& Grzegorz Turnau                     & 623.00                                                     \\
	 Mira Kubasińska \& After Blues                             & 610.00                                                     \\
	 Sabaton                                                   & 608.00                                                   \\
	 Yugopolis 2 / Maciej Maleńczuk                            & 607.00                                                   \\
	 Brodka \& A\_GIM                                            & 602.00                                                       \\
	 Pati Yang                                                 & 595.00                                                   \\
	 Sting feat. Cheb Mami                                     & 591.00                                                   \\
	 Lady Gaga \& Bradley Cooper                                & 567.00                                                     \\
	 Krzysztof Kiljański \& Kayah                               & 565.00                                                     \\
	 Ania Rusowicz \& Adam Nowak                                & 561.00                                                     \\
	 Marek Grechuta \& Myslovitz                                & 560.00                                                     \\
	 Santana feat. Rob Thomas                                  & 558.00                                                   \\
\end{tabular}


    
    Zaobserwowałem, że początek powyższej tabeli jest stosunkowo pełen
muzycznych kolaboracji, np. ``Kazik \& Edyta Bartosiewicz'', ``Gotye
feat. Kimbra''. Czy jest to ogólna zasada? Czy pojedyncze współprace
między artystami to klucz do sukcesu? Sprawdźmy zatem, jak wygląda to
dla całej tabeli.

\hypertarget{test}{%
\paragraph{Test}\label{test}}

Dla uproszczenia przyjmiemy, że muzyczne kolaboracje mają w nazwie znak
``\&'' bądź słowo ``feat.'' czyli ``featuring''. Ważne jest, że nie jest
to metoda perfekcyjna. Często nie jest to jednorazowa współpraca jak w
przypadku Nicka Cave'a i zespołu The Bad Seeds, którzy tworzą muzykę
wspólnie już przez kilka dekad. Nie wliczamy jednak np. Męskiego Grania
2018, które jest oczywiście, zbiorem twórców tworzących zazwyczaj osobno

    \begin{tcolorbox}[breakable, size=fbox, boxrule=1pt, pad at break*=1mm,colback=cellbackground, colframe=cellborder]
\prompt{In}{incolor}{192}{\boxspacing}
\begin{Verbatim}[commandchars=\\\{\}]
\PY{n}{kolaboracjeTabela} \PY{o}{\PYZlt{}\PYZhy{}}
    ś\PY{n}{rPktNaArtyste} \PY{o}{\PYZpc{}\PYZgt{}\PYZpc{}}
    \PY{n+nf}{filter}\PY{p}{(}\PY{n+nf}{str\PYZus{}detect}\PY{p}{(}\PY{n}{Artysta}\PY{p}{,} \PY{l+s}{\PYZsq{}}\PY{l+s}{\PYZam{}|feat.\PYZsq{}}\PY{p}{)}\PY{p}{)}

\PY{n}{calosciowo} \PY{o}{\PYZlt{}\PYZhy{}}
    ś\PY{n}{rPktNaArtyste} \PY{o}{\PYZpc{}\PYZgt{}\PYZpc{}}
    \PY{n+nf}{summarise}\PY{p}{(}
        \PY{n}{nazwa} \PY{o}{=} \PY{l+s}{\PYZdq{}}\PY{l+s}{Wszyscy artyści\PYZdq{}}\PY{p}{,}
        \PY{n}{liczba\PYZus{}utworów}\PY{o}{=} \PY{n+nf}{n}\PY{p}{(}\PY{p}{)}\PY{p}{,}
        ś\PY{n}{rednia\PYZus{}liczba\PYZus{}punktów} \PY{o}{=} \PY{n+nf}{mean}\PY{p}{(}Ś\PY{n}{redniaPunktówNaUtwór}\PY{p}{)}\PY{p}{,}
        \PY{n}{mediana} \PY{o}{=} \PY{n+nf}{median}\PY{p}{(}Ś\PY{n}{redniaPunktówNaUtwór}\PY{p}{)}\PY{p}{,}
        \PY{n}{odchylenie\PYZus{}standardowe} \PY{o}{=} \PY{n+nf}{sd}\PY{p}{(}Ś\PY{n}{redniaPunktówNaUtwór}\PY{p}{)}
        \PY{p}{)}

\PY{n}{kolaboracje} \PY{o}{\PYZlt{}\PYZhy{}}     
    \PY{n}{kolaboracjeTabela} \PY{o}{\PYZpc{}\PYZgt{}\PYZpc{}}
    \PY{n+nf}{summarise}\PY{p}{(}
        \PY{n}{nazwa} \PY{o}{=} \PY{l+s}{\PYZdq{}}\PY{l+s}{Kolaboracje\PYZdq{}}\PY{p}{,}
        \PY{n}{liczba\PYZus{}utworów} \PY{o}{=} \PY{n+nf}{n}\PY{p}{(}\PY{p}{)}\PY{p}{,}
        ś\PY{n}{rednia\PYZus{}liczba\PYZus{}punktów} \PY{o}{=} \PY{n+nf}{mean}\PY{p}{(}Ś\PY{n}{redniaPunktówNaUtwór}\PY{p}{)}\PY{p}{,}
        \PY{n}{mediana} \PY{o}{=} \PY{n+nf}{median}\PY{p}{(}Ś\PY{n}{redniaPunktówNaUtwór}\PY{p}{)}\PY{p}{,}
        \PY{n}{odchylenie\PYZus{}standardowe} \PY{o}{=} \PY{n+nf}{sd}\PY{p}{(}Ś\PY{n}{redniaPunktówNaUtwór}\PY{p}{)}
        \PY{p}{)}


\PY{n+nf}{rbind}\PY{p}{(}\PY{n}{calosciowo}\PY{p}{,} \PY{n}{kolaboracje}\PY{p}{)}

ś\PY{n}{rPktNaArtyste} \PY{o}{\PYZpc{}\PYZgt{}\PYZpc{}}    
\PY{n+nf}{filter}\PY{p}{(}\PY{n+nf}{str\PYZus{}detect}\PY{p}{(}\PY{n}{Artysta}\PY{p}{,} \PY{l+s}{\PYZsq{}}\PY{l+s}{\PYZam{}|feat.\PYZsq{}}\PY{p}{)}\PY{p}{)} \PY{o}{\PYZpc{}\PYZgt{}\PYZpc{}}
\PY{n+nf}{ggplot}\PY{p}{(}\PY{n+nf}{aes}\PY{p}{(}Ś\PY{n}{redniaPunktówNaUtwór}\PY{p}{)}\PY{p}{)} \PY{o}{+}
    \PY{n+nf}{theme\PYZus{}bw}\PY{p}{(}\PY{p}{)}\PY{o}{+}
    \PY{n+nf}{geom\PYZus{}bar}\PY{p}{(} \PY{n}{stat}\PY{o}{=}\PY{l+s}{\PYZdq{}}\PY{l+s}{bin\PYZdq{}}\PY{p}{,} \PY{n}{binwidth}\PY{o}{=}\PY{l+m}{10}\PY{p}{,} \PY{n}{fill}\PY{o}{=}\PY{l+s}{\PYZsq{}}\PY{l+s}{lightgreen\PYZsq{}}\PY{p}{)} \PY{o}{+} \PY{n+nf}{ylab}\PY{p}{(}\PY{l+s}{\PYZdq{}}\PY{l+s}{Liczba utworów\PYZdq{}}\PY{p}{)}\PY{o}{+}
    \PY{n+nf}{ggtitle}\PY{p}{(}\PY{l+s}{\PYZdq{}}\PY{l+s}{Liczba artystów z daną średnią punktów\PYZdq{}}\PY{p}{)}
\end{Verbatim}
\end{tcolorbox}

    \begin{tabular}{r|lllll}
 nazwa & liczba\_utworów & średnia\_liczba\_punktów & mediana & odchylenie\_standardowe\\
\hline
	 Wszyscy artyści & 1826            & 111.6308        & 73.30952        & 122.6950       \\
	 Kolaboracje     &  371            & 137.1124        & 68.00000        & 164.3714       \\
\end{tabular}


    
    \begin{center}
    \adjustimage{max size={0.9\linewidth}{0.9\paperheight}}{Analiza_Listy_Przebojow_Trojki_files/Analiza_Listy_Przebojow_Trojki_33_1.png}
    \end{center}
    { \hspace*{\fill} \\}
    
    Powyższa tabelka pokazuje, że początkowa hipoteza na bazie fragmentu
danych, nie była poprawna. Utwory powstałe jako muzyczne kolaboracje
mają co prawda wyższą średnią sumarycznych punktów na utwór, jednak
niższą medianę. Oznacza to, że w porównaniu z ogółem utworów, omawiane
piosenki mogą w znaczący sposób ``wyprzedzić'' resztę, jednak jest to
jeszcze bardziej niepewne niż w normalnym przypadku, na co wskazuje
również większe odchylenie standardowe.

Sprawdźmy, czy istnieje korelacja między liczbą piosenek w zestawieniu a
średnią zdobytą liczbą punktów na utwór dla utworów muzycznych
kolaboracji

    \begin{tcolorbox}[breakable, size=fbox, boxrule=1pt, pad at break*=1mm,colback=cellbackground, colframe=cellborder]
\prompt{In}{incolor}{191}{\boxspacing}
\begin{Verbatim}[commandchars=\\\{\}]
\PY{n}{kolaboracjeLiczbaPiosenek} \PY{o}{\PYZlt{}\PYZhy{}} \PY{p}{(}\PY{n}{liczbaPiosenekNaArtyste} \PY{o}{\PYZpc{}\PYZgt{}\PYZpc{}} \PY{n+nf}{filter}\PY{p}{(}\PY{n+nf}{str\PYZus{}detect}\PY{p}{(}\PY{n}{Artysta}\PY{p}{,} \PY{l+s}{\PYZsq{}}\PY{l+s}{\PYZam{}|feat.\PYZsq{}}\PY{p}{)}\PY{p}{)}\PY{p}{)}
\PY{n}{kolaboracjeŚrednia} \PY{o}{\PYZlt{}\PYZhy{}} \PY{p}{(}ś\PY{n}{rPktNaArtyste} \PY{o}{\PYZpc{}\PYZgt{}\PYZpc{}} \PY{n+nf}{filter}\PY{p}{(}\PY{n+nf}{str\PYZus{}detect}\PY{p}{(}\PY{n}{Artysta}\PY{p}{,} \PY{l+s}{\PYZsq{}}\PY{l+s}{\PYZam{}|feat.\PYZsq{}}\PY{p}{)}\PY{p}{)}\PY{p}{)}                      

\PY{n}{joinedTable} \PY{o}{\PYZlt{}\PYZhy{}} \PY{n+nf}{left\PYZus{}join}\PY{p}{(}\PY{n}{kolaboracjeLiczbaPiosenek}\PY{p}{,} \PY{n}{kolaboracjeŚrednia}\PY{p}{,} \PY{n}{by}\PY{o}{=}\PY{l+s}{\PYZdq{}}\PY{l+s}{Artysta\PYZdq{}}\PY{p}{)}
\PY{n}{korelacja} \PY{o}{\PYZlt{}\PYZhy{}} \PY{n+nf}{cor}\PY{p}{(}\PY{n}{joinedTable}\PY{o}{\PYZdl{}}\PY{n}{Liczba.utworów.na.liście}\PY{p}{,} \PY{n}{joinedTable}\PY{o}{\PYZdl{}}Ś\PY{n}{redniaPunktówNaUtwór}\PY{p}{)}

    \PY{n}{joinedTable} \PY{o}{\PYZpc{}\PYZgt{}\PYZpc{}}
    \PY{n+nf}{ggplot}\PY{p}{(}\PY{n+nf}{aes}\PY{p}{(}\PY{n}{x} \PY{o}{=} \PY{n}{Liczba.utworów.na.liście}\PY{p}{,} \PY{n}{y}\PY{o}{=}Ś\PY{n}{redniaPunktówNaUtwór}\PY{p}{)}\PY{p}{)} \PY{o}{+}
    \PY{n+nf}{geom\PYZus{}point}\PY{p}{(}\PY{p}{)} \PY{o}{+}
    \PY{n+nf}{stat\PYZus{}smooth}\PY{p}{(}\PY{n}{method} \PY{o}{=} \PY{n}{lm}\PY{p}{,} \PY{n}{formula}\PY{o}{=}\PY{n}{y}\PY{o}{\PYZti{}}\PY{n}{x}\PY{p}{)} \PY{o}{+}  \PY{c+c1}{\PYZsh{}poszukiwanie regresji liniowej}
    \PY{n+nf}{theme\PYZus{}bw}\PY{p}{(}\PY{p}{)} \PY{o}{+}
    \PY{n+nf}{ggtitle}\PY{p}{(}\PY{l+s}{\PYZdq{}}\PY{l+s}{Muzyczne kolaboracje a \PYZbs{}\PYZdq{}jakość\PYZbs{}\PYZdq{} utworu\PYZbs{}nKorelacja:\PYZdq{}}\PY{p}{,} \PY{n}{korelacja}\PY{p}{)}
\end{Verbatim}
\end{tcolorbox}

    \begin{center}
    \adjustimage{max size={0.9\linewidth}{0.9\paperheight}}{Analiza_Listy_Przebojow_Trojki_files/Analiza_Listy_Przebojow_Trojki_35_0.png}
    \end{center}
    { \hspace*{\fill} \\}
    
    Jak widać - teraz korelacja jest jeszcze mniejsza. Oczywiście można
również sprawdzić, czy kolaboracje częstych ``bywaczy'' na liście (jak
np. ``Kazik \& Edyta Bartosiewicz'') zapewniają więcej punktów niż
typowy efekt współpracy między twórcami, jednak jest to już historia na
inny czas

    \hypertarget{czy-wysoki-debiut-zapewnia-dux142ugi-czas-na-liux15bcie}{%
\subsubsection{Czy wysoki debiut zapewnia długi czas na
liście?}\label{czy-wysoki-debiut-zapewnia-dux142ugi-czas-na-liux15bcie}}

\hypertarget{gux142uxf3wna-myux15bl}{%
\paragraph{Główna myśl}\label{gux142uxf3wna-myux15bl}}

Możemy również zadać sobie pytanie - czy utwory, które zaczynają ``z
najwyższej belki'', stają się przebojami na długo? Czy istnieje i jeżeli
tak, to jak wygląda zależność między wysokim debiutem a liczbą tygodni
spędzonych na Liście Przebojów Trójki

    \begin{tcolorbox}[breakable, size=fbox, boxrule=1pt, pad at break*=1mm,colback=cellbackground, colframe=cellborder]
\prompt{In}{incolor}{196}{\boxspacing}
\begin{Verbatim}[commandchars=\\\{\}]
\PY{n}{res} \PY{o}{\PYZlt{}\PYZhy{}} 
    \PY{n}{records} \PY{o}{\PYZpc{}\PYZgt{}\PYZpc{}}
    \PY{n+nf}{group\PYZus{}by}\PY{p}{(}\PY{n}{Tytuł}\PY{p}{,} \PY{n}{Artysta}\PY{p}{)} \PY{o}{\PYZpc{}\PYZgt{}\PYZpc{}}
    \PY{n+nf}{arrange}\PY{p}{(}\PY{n}{Tytuł}\PY{p}{,} \PY{n}{Artysta}\PY{p}{)} \PY{o}{\PYZpc{}\PYZgt{}\PYZpc{}}
    \PY{n+nf}{mutate}\PY{p}{(}\PY{n}{TygNaLiście}\PY{o}{=}\PY{n+nf}{n}\PY{p}{(}\PY{p}{)}\PY{p}{)} \PY{o}{\PYZpc{}\PYZgt{}\PYZpc{}}
    \PY{n+nf}{filter}\PY{p}{(}\PY{n+nf}{row\PYZus{}number}\PY{p}{(}\PY{p}{)}\PY{o}{==}\PY{l+m}{1}\PY{p}{)} \PY{o}{\PYZpc{}\PYZgt{}\PYZpc{}}
    \PY{n+nf}{select}\PY{p}{(}\PY{n}{Tytuł}\PY{p}{,} \PY{n}{Artysta}\PY{p}{,} \PY{n}{Pozycja}\PY{p}{,} \PY{n}{TygNaLiście}\PY{p}{)} \PY{o}{\PYZpc{}\PYZgt{}\PYZpc{}}
    \PY{n+nf}{rename}\PY{p}{(}\PY{n}{PozycjaDebiutu} \PY{o}{=} \PY{n}{Pozycja}\PY{p}{)} \PY{o}{\PYZpc{}\PYZgt{}\PYZpc{}}
    \PY{n+nf}{ungroup}\PY{p}{(}\PY{p}{)}

\PY{n}{cor} \PY{o}{\PYZlt{}\PYZhy{}} \PY{n+nf}{cor}\PY{p}{(}\PY{n}{res}\PY{o}{\PYZdl{}}\PY{n}{PozycjaDebiutu}\PY{p}{,} \PY{n}{res}\PY{o}{\PYZdl{}}\PY{n}{TygNaLiście}\PY{p}{)}
\PY{n+nf}{cor.test}\PY{p}{(}\PY{n}{res}\PY{o}{\PYZdl{}}\PY{n}{PozycjaDebiutu}\PY{p}{,} \PY{n}{res}\PY{o}{\PYZdl{}}\PY{n}{TygNaLiście}\PY{p}{)}

\PY{n}{pktDoNarysowania} \PY{o}{\PYZlt{}\PYZhy{}} 
    \PY{n}{res} \PY{o}{\PYZpc{}\PYZgt{}\PYZpc{}}
    \PY{n+nf}{group\PYZus{}by}\PY{p}{(}\PY{n}{PozycjaDebiutu}\PY{p}{,} \PY{n}{TygNaLiście}\PY{p}{)} \PY{o}{\PYZpc{}\PYZgt{}\PYZpc{}}
    \PY{n+nf}{summarise}\PY{p}{(}\PY{n}{Powtórzenia} \PY{o}{=} \PY{n+nf}{n}\PY{p}{(}\PY{p}{)}\PY{p}{)}

\PY{n}{pktDoNarysowania} \PY{o}{\PYZpc{}\PYZgt{}\PYZpc{}}
    \PY{n+nf}{ggplot}\PY{p}{(}\PY{n+nf}{aes}\PY{p}{(}\PY{n}{x} \PY{o}{=} \PY{n}{PozycjaDebiutu}\PY{p}{,} \PY{n}{y}\PY{o}{=}\PY{n}{TygNaLiście}\PY{p}{,} \PY{n}{color}\PY{o}{=}\PY{n}{Powtórzenia}\PY{p}{)}\PY{p}{)} \PY{o}{+}
    \PY{n+nf}{geom\PYZus{}point}\PY{p}{(}\PY{p}{)} \PY{o}{+}
    \PY{n+nf}{scale\PYZus{}color\PYZus{}gradient}\PY{p}{(}\PY{n}{low}\PY{o}{=}\PY{l+s}{\PYZdq{}}\PY{l+s}{blue\PYZdq{}}\PY{p}{,} \PY{n}{high}\PY{o}{=}\PY{l+s}{\PYZdq{}}\PY{l+s}{red\PYZdq{}}\PY{p}{)} \PY{o}{+}
    \PY{n+nf}{geom\PYZus{}smooth}\PY{p}{(}\PY{n}{method}\PY{o}{=}\PY{n}{lm}\PY{p}{,} \PY{n}{formula}\PY{o}{=}\PY{n}{y}\PY{o}{\PYZti{}}\PY{n}{x}\PY{p}{)} \PY{o}{+}
    \PY{n+nf}{ggtitle}\PY{p}{(}\PY{l+s}{\PYZdq{}}\PY{l+s}{Liczba tyg. na liście w zależności od pozycji debiutu\PYZbs{}nKorelacja:\PYZdq{}} \PY{p}{,} \PY{n}{cor}\PY{p}{)}
\end{Verbatim}
\end{tcolorbox}

    
    \begin{verbatim}

	Pearson's product-moment correlation

data:  res$PozycjaDebiutu and res$TygNaLiście
t = -29.925, df = 6223, p-value < 2.2e-16
alternative hypothesis: true correlation is not equal to 0
95 percent confidence interval:
 -0.3762059 -0.3327679
sample estimates:
       cor 
-0.3546783 

    \end{verbatim}

    
    \begin{center}
    \adjustimage{max size={0.9\linewidth}{0.9\paperheight}}{Analiza_Listy_Przebojow_Trojki_files/Analiza_Listy_Przebojow_Trojki_38_1.png}
    \end{center}
    { \hspace*{\fill} \\}
    
    Na bazie wyliczonej korelacji oraz testu Pearsona można stwierdzić, że
istnieje pewna ujemna korelacja między wysokim debiutem a długim
występowaniem utworu na liście przebojów. \#\#\#\# Wyjątki od reguły
Należy jednak zauważyć, że jest to stosunkowo mała korelacja, od której
zdarzają się znaczne wyjątki. Tak np. wyglądała droga na szczyt utworu
\emph{Again} zespołu \emph{Archive}

    \begin{tcolorbox}[breakable, size=fbox, boxrule=1pt, pad at break*=1mm,colback=cellbackground, colframe=cellborder]
\prompt{In}{incolor}{197}{\boxspacing}
\begin{Verbatim}[commandchars=\\\{\}]
\PY{n}{records} \PY{o}{\PYZpc{}\PYZgt{}\PYZpc{}}
    \PY{n+nf}{filter}\PY{p}{(}\PY{n}{Tytuł}\PY{o}{==}\PY{l+s}{\PYZdq{}}\PY{l+s}{Again\PYZdq{}}\PY{p}{,} \PY{n}{Artysta}\PY{o}{==}\PY{l+s}{\PYZdq{}}\PY{l+s}{Archive\PYZdq{}}\PY{p}{)} \PY{o}{\PYZpc{}\PYZgt{}\PYZpc{}}
    \PY{n+nf}{ggplot}\PY{p}{(}\PY{p}{)} \PY{o}{+}
    \PY{n+nf}{geom\PYZus{}line}\PY{p}{(}\PY{n+nf}{aes}\PY{p}{(}\PY{n}{Nr.notowania}\PY{p}{,} \PY{n}{Pozycja}\PY{p}{)}\PY{p}{)} \PY{o}{+} \PY{n+nf}{theme\PYZus{}bw}\PY{p}{(}\PY{p}{)} \PY{o}{+} \PY{n+nf}{scale\PYZus{}y\PYZus{}reverse}\PY{p}{(}\PY{p}{)} \PY{o}{+}
    \PY{n+nf}{geom\PYZus{}point}\PY{p}{(}\PY{n+nf}{aes}\PY{p}{(}\PY{n}{Nr.notowania}\PY{p}{,} \PY{n}{Pozycja}\PY{p}{,}\PY{n}{color}\PY{o}{=}\PY{n}{Pozycja}\PY{p}{)}\PY{p}{,} \PY{n}{size}\PY{o}{=}\PY{l+m}{3}\PY{p}{)} \PY{o}{+}
    \PY{n+nf}{scale\PYZus{}color\PYZus{}gradient}\PY{p}{(}\PY{n}{low}\PY{o}{=}\PY{l+s}{\PYZdq{}}\PY{l+s}{red\PYZdq{}}\PY{p}{,} \PY{n}{high}\PY{o}{=}\PY{l+s}{\PYZdq{}}\PY{l+s}{blue\PYZdq{}}\PY{p}{)} \PY{o}{+}
    \PY{n+nf}{ggtitle}\PY{p}{(}\PY{l+s}{\PYZdq{}}\PY{l+s}{Pozycje utworu \PYZbs{}\PYZdq{}Again\PYZbs{}\PYZdq{} zespołu \PYZbs{}\PYZdq{}Archive\PYZbs{}\PYZdq{}\PYZdq{}}\PY{p}{)}
\end{Verbatim}
\end{tcolorbox}

    \begin{center}
    \adjustimage{max size={0.9\linewidth}{0.9\paperheight}}{Analiza_Listy_Przebojow_Trojki_files/Analiza_Listy_Przebojow_Trojki_40_0.png}
    \end{center}
    { \hspace*{\fill} \\}
    
    Jak widać - wysoka belka nie zawsze oznacza wysoki skok

    \hypertarget{efekt-mux119skiego-grania-czyli-o-wpux142ywie-muzycnej-inicjatywy-ux17cywca-na-listux119-przebojuxf3w}{%
\subsubsection{\texorpdfstring{\emph{Efekt Męskiego Grania}, czyli o
wpływie muzycnej inicjatywy Żywca na Listę
Przebojów}{Efekt Męskiego Grania, czyli o wpływie muzycnej inicjatywy Żywca na Listę Przebojów}}\label{efekt-mux119skiego-grania-czyli-o-wpux142ywie-muzycnej-inicjatywy-ux17cywca-na-listux119-przebojuxf3w}}

\hypertarget{wstux119p}{%
\paragraph{Wstęp}\label{wstux119p}}

Na koniec sprawdzimy, czy i jakich artystów ``wypromowała'' trasa
``Męskiego Grania''. ``Męskie granie'' to zainicjowana i tworzona przez
markę Żywiec trasa koncertowa, skupiająca sporą grupę polskich artystów
związanych ze sceną rockową i hiphopową. Pierwsze koncerty rozpoczęły
się w 2010; promujący tę trasę singiel \emph{Wszyscy muzyce to
wojownicy} przez 24 tygodnie znajdował się na Liście Przebojów Trójki,
uzyskując 482 punkty (co pokazuje poniższa tabela).

Przez następne lata skala projektu rosła, a wraz z nią pozycje
``Męskiego grania'' na Liście, co widać na poniższym diagramie,
przedstawiającym sumy punktów uzyskanych przez utwory wydane pod szyldem
``Męskiego Grania'' w kolejnych latach.

    \begin{tcolorbox}[breakable, size=fbox, boxrule=1pt, pad at break*=1mm,colback=cellbackground, colframe=cellborder]
\prompt{In}{incolor}{199}{\boxspacing}
\begin{Verbatim}[commandchars=\\\{\}]
\PY{n}{records} \PY{o}{\PYZpc{}\PYZgt{}\PYZpc{}}
    \PY{n+nf}{filter}\PY{p}{(}\PY{n+nf}{str\PYZus{}detect}\PY{p}{(}\PY{n}{Tytuł}\PY{p}{,} \PY{l+s}{\PYZdq{}}\PY{l+s}{Wszyscy muzycy to wojownicy\PYZdq{}}\PY{p}{)}\PY{p}{)} \PY{o}{\PYZpc{}\PYZgt{}\PYZpc{}}
    \PY{n+nf}{group\PYZus{}by}\PY{p}{(}\PY{n}{Tytuł}\PY{p}{)} \PY{o}{\PYZpc{}\PYZgt{}\PYZpc{}}
    \PY{n+nf}{summarise}\PY{p}{(}\PY{n}{Tyg\PYZus{}na\PYZus{}liście} \PY{o}{=} \PY{n+nf}{n}\PY{p}{(}\PY{p}{)}\PY{p}{,}
              \PY{n}{Suma\PYZus{}Punktów} \PY{o}{=} \PY{n+nf}{sum}\PY{p}{(}\PY{n}{Punkty}\PY{p}{)}\PY{p}{)}

\PY{n}{records} \PY{o}{\PYZpc{}\PYZgt{}\PYZpc{}} 
    \PY{n+nf}{filter}\PY{p}{(}\PY{n+nf}{str\PYZus{}detect}\PY{p}{(}\PY{n}{Artysta}\PY{p}{,} \PY{l+s}{\PYZsq{}}\PY{l+s}{Męskie Granie\PYZsq{}}\PY{p}{)}\PY{p}{)} \PY{o}{\PYZpc{}\PYZgt{}\PYZpc{}}
    \PY{n+nf}{group\PYZus{}by}\PY{p}{(}\PY{n}{Rok.notowania}\PY{p}{)} \PY{o}{\PYZpc{}\PYZgt{}\PYZpc{}}
    \PY{n+nf}{summarise}\PY{p}{(}\PY{n}{Suma\PYZus{}Punktów} \PY{o}{=} \PY{n+nf}{sum}\PY{p}{(}\PY{n}{Punkty}\PY{p}{)}\PY{p}{)} \PY{o}{\PYZpc{}\PYZgt{}\PYZpc{}}
    \PY{n+nf}{ggplot}\PY{p}{(}\PY{n+nf}{aes}\PY{p}{(}\PY{n}{Rok.notowania}\PY{p}{,} \PY{n}{Suma\PYZus{}Punktów}\PY{p}{)}\PY{p}{)} \PY{o}{+}
    \PY{n+nf}{theme\PYZus{}bw}\PY{p}{(}\PY{p}{)} \PY{o}{+}
    \PY{n+nf}{geom\PYZus{}bar}\PY{p}{(}\PY{n}{stat}\PY{o}{=}\PY{l+s}{\PYZdq{}}\PY{l+s}{identity\PYZdq{}}\PY{p}{,} \PY{n}{fill}\PY{o}{=}\PY{l+s}{\PYZsq{}}\PY{l+s}{\PYZsh{}FF6666\PYZsq{}}\PY{p}{)} \PY{o}{+} \PY{n+nf}{scale\PYZus{}x\PYZus{}continuous}\PY{p}{(}\PY{n}{breaks} \PY{o}{=} \PY{n+nf}{seq}\PY{p}{(}\PY{l+m}{2010}\PY{p}{,} \PY{l+m}{2020}\PY{p}{,} \PY{n}{by} \PY{o}{=} \PY{l+m}{1}\PY{p}{)}\PY{p}{)} \PY{o}{+}
    \PY{n+nf}{ggtitle}\PY{p}{(}\PY{l+s}{\PYZdq{}}\PY{l+s}{Liczba punktów utworów z nazwą \PYZbs{}\PYZdq{}Męskie Granie\PYZbs{}\PYZdq{} w nazwie w danym roku\PYZdq{}}\PY{p}{)}
\end{Verbatim}
\end{tcolorbox}

    \begin{tabular}{r|lll}
 Tytuł & Tyg\_na\_liście & Suma\_Punktów\\
\hline
	 Wszyscy muzycy to wojownicy & 24                          & 482                        \\
\end{tabular}


    
    \begin{center}
    \adjustimage{max size={0.9\linewidth}{0.9\paperheight}}{Analiza_Listy_Przebojow_Trojki_files/Analiza_Listy_Przebojow_Trojki_43_1.png}
    \end{center}
    { \hspace*{\fill} \\}
    
    O ile sam wzrost popularności Męskiego Grania nie ulega wątpliwości, tak
możemy sprawdzić, jaki wpływ miał on na pozycje niektórych artystów w
nim występujących. W tym celu sprawdzimy kumulowane punkty na
przestrzeni lat danych artystów biorących udział w projekcie.

\hypertarget{sposuxf3b-wykonania-testu}{%
\paragraph{Sposób wykonania testu}\label{sposuxf3b-wykonania-testu}}

Do punktów danego artysty doliczamy wszystkie jego kolaboracje z innyi
artystami oprócz Męskiego Grania (czyli jeśli ich nazwisko pojawiło się
w rubryce ``Artysta''). Widoczne na wykresie wertykalne linie to daty
pierwszych koncertów danej edycji Męskiego Grania dla odpowiednio 2010,
2014, 2017 i 2019 roku. Wybrano je, ponieważ wtedy popularność utworów
Męskiego Grania (wg powyższego wykresu) była największa.

Sami ``artyści męskiego grania'' zostali wybrani arbitralnie. Niestety
nie znalazłem bazy danych, która zawierałaby wszystkich artystów
zaangażowanych w tenże projekt. Starałem się zatem wymienić czołowe
nazwiska, które towarzyszyły niejednokrotnie zarówno Męskiemu Graniu,
jak i omawianej liście.

\hypertarget{test}{%
\paragraph{Test}\label{test}}

    \begin{tcolorbox}[breakable, size=fbox, boxrule=1pt, pad at break*=1mm,colback=cellbackground, colframe=cellborder]
\prompt{In}{incolor}{200}{\boxspacing}
\begin{Verbatim}[commandchars=\\\{\}]
\PY{n}{artysci\PYZus{}meskiego\PYZus{}grania} \PY{o}{\PYZlt{}\PYZhy{}} \PY{n+nf}{c}\PY{p}{(}\PY{l+s}{\PYZdq{}}\PY{l+s}{O.S.T.R.\PYZdq{}}\PY{p}{,} \PY{l+s}{\PYZdq{}}\PY{l+s}{Lao Che\PYZdq{}}\PY{p}{,} \PY{l+s}{\PYZdq{}}\PY{l+s}{Brodka\PYZdq{}}\PY{p}{,} \PY{l+s}{\PYZdq{}}\PY{l+s}{Zawiałow\PYZdq{}}\PY{p}{,} \PY{l+s}{\PYZdq{}}\PY{l+s}{Nosowska\PYZdq{}}\PY{p}{,} \PY{l+s}{\PYZdq{}}\PY{l+s}{Waglewski\PYZdq{}}\PY{p}{,} \PY{l+s}{\PYZdq{}}\PY{l+s}{Organek\PYZdq{}}\PY{p}{,} \PY{l+s}{\PYZdq{}}\PY{l+s}{Podsiadło\PYZdq{}}\PY{p}{,} \PY{l+s}{\PYZdq{}}\PY{l+s}{Zalewski\PYZdq{}}\PY{p}{,} \PY{l+s}{\PYZdq{}}\PY{l+s}{Maleńczuk\PYZdq{}}\PY{p}{,} \PY{l+s}{\PYZdq{}}\PY{l+s}{Smolik\PYZdq{}}\PY{p}{)}

\PY{n}{res} \PY{o}{\PYZlt{}\PYZhy{}} \PY{n+nf}{data.frame}\PY{p}{(}\PY{n}{Nr.notowania}\PY{o}{=}\PY{n+nf}{integer}\PY{p}{(}\PY{p}{)}\PY{p}{,}
                  \PY{n}{Rok.notowania}\PY{o}{=}\PY{n+nf}{integer}\PY{p}{(}\PY{p}{)}\PY{p}{,}
                  \PY{n}{Artysta}\PY{o}{=}\PY{n+nf}{character}\PY{p}{(}\PY{p}{)}\PY{p}{,}
                  \PY{n}{KumulowanePunkty}\PY{o}{=}\PY{n+nf}{integer}\PY{p}{(}\PY{p}{)}\PY{p}{)}

\PY{n+nf}{for }\PY{p}{(}\PY{n}{artystaMesGran} \PY{n}{in} \PY{n}{artysci\PYZus{}meskiego\PYZus{}grania}\PY{p}{)}\PY{p}{\PYZob{}}
    \PY{n}{tmp} \PY{o}{\PYZlt{}\PYZhy{}}
        \PY{n}{records} \PY{o}{\PYZpc{}\PYZgt{}\PYZpc{}}
        \PY{n+nf}{filter}\PY{p}{(}\PY{n+nf}{str\PYZus{}detect}\PY{p}{(}\PY{n}{Artysta}\PY{p}{,} \PY{n}{artystaMesGran}\PY{p}{)}\PY{p}{)}\PY{o}{\PYZpc{}\PYZgt{}\PYZpc{}}
        \PY{n+nf}{mutate}\PY{p}{(}\PY{n}{KumulowanePunkty}\PY{o}{=}\PY{n+nf}{cumsum}\PY{p}{(}\PY{n}{Punkty}\PY{p}{)}\PY{p}{)} \PY{o}{\PYZpc{}\PYZgt{}\PYZpc{}}
        \PY{n+nf}{mutate}\PY{p}{(}\PY{n}{Artysta}\PY{o}{=}\PY{n}{artystaMesGran}\PY{p}{)} \PY{o}{\PYZpc{}\PYZgt{}\PYZpc{}}
        \PY{n+nf}{select}\PY{p}{(}\PY{n}{Nr.notowania}\PY{p}{,} \PY{n}{Rok.notowania}\PY{p}{,} \PY{n}{Artysta}\PY{p}{,} \PY{n}{KumulowanePunkty}\PY{p}{)}
    
    \PY{n}{res} \PY{o}{\PYZlt{}\PYZhy{}} \PY{n+nf}{rbind}\PY{p}{(}\PY{n}{res}\PY{p}{,} \PY{n}{tmp}\PY{p}{)}
\PY{p}{\PYZcb{}}

\PY{n}{res} \PY{o}{\PYZlt{}\PYZhy{}} \PY{n}{res} \PY{o}{\PYZpc{}\PYZgt{}\PYZpc{}} 
    \PY{n+nf}{filter}\PY{p}{(}\PY{n}{Nr.notowania} \PY{o}{\PYZgt{}} \PY{l+m}{750}\PY{p}{)}  \PY{c+c1}{\PYZsh{}zbyt \PYZdq{}stare\PYZdq{} notowania}

\PY{n}{nr\PYZus{}notowania\PYZus{}m\PYZus{}gran} \PY{o}{=} \PY{n+nf}{c}\PY{p}{(}\PY{l+m}{1486}\PY{p}{,}   \PY{c+c1}{\PYZsh{}1. koncert Męskiego Grania}
                        \PY{l+m}{1690}\PY{p}{,}   \PY{c+c1}{\PYZsh{}1. koncert MG 2014}
                        \PY{l+m}{1849}\PY{p}{,}   \PY{c+c1}{\PYZsh{}1. koncert MG 2017}
                        \PY{l+m}{1954}\PY{p}{)}   \PY{c+c1}{\PYZsh{}1. koncert MG 2019}
\PY{n}{res} \PY{o}{\PYZpc{}\PYZgt{}\PYZpc{}} 
    \PY{n+nf}{ggplot}\PY{p}{(}\PY{p}{)} \PY{o}{+} \PY{n+nf}{ggtitle}\PY{p}{(}\PY{l+s}{\PYZdq{}}\PY{l+s}{Kumulowane punkty artystów \PYZbs{}\PYZdq{}Męskiego Grania \PYZbs{}\PYZdq{}\PYZdq{}}\PY{p}{)} \PY{o}{+}
    \PY{n+nf}{geom\PYZus{}line}\PY{p}{(}\PY{n+nf}{aes}\PY{p}{(}\PY{n}{Nr.notowania}\PY{p}{,} \PY{n}{KumulowanePunkty}\PY{p}{,} \PY{n}{color}\PY{o}{=}\PY{n}{Artysta}\PY{p}{)}\PY{p}{)} \PY{o}{+} \PY{n+nf}{theme\PYZus{}bw}\PY{p}{(}\PY{p}{)} \PY{o}{+} \PY{n+nf}{geom\PYZus{}vline}\PY{p}{(}\PY{n}{xintercept}\PY{o}{=}\PY{n}{nr\PYZus{}notowania\PYZus{}m\PYZus{}gran}\PY{p}{,} \PY{n}{linetype}\PY{o}{=}\PY{l+s}{\PYZdq{}}\PY{l+s}{dotted\PYZdq{}}\PY{p}{)}
\end{Verbatim}
\end{tcolorbox}

    \begin{center}
    \adjustimage{max size={0.9\linewidth}{0.9\paperheight}}{Analiza_Listy_Przebojow_Trojki_files/Analiza_Listy_Przebojow_Trojki_45_0.png}
    \end{center}
    { \hspace*{\fill} \\}
    
    \hypertarget{prosta-analiza}{%
\paragraph{Prosta analiza}\label{prosta-analiza}}

Pomimo dosyć małej liczby analizowanych twórców, można ich podzielić na
pomniejsze grupy artystów: 1. na których pozycje w liście przebojów
znacznie nie wpłynęła 2. którzy tworzyli wcześniej, lecz dzięki Męskiemu
Graniu lepiej radzili sobie na liście 3. którzy zdobyli sporą
popularność po powstaniu Męskiego Grania

Analizę pierwszej grupy pominiemy, do których z wybranej próbki możemy
zaliczyć jedynie Andrzeja Smolika, który jako jedyny radził sobie lepiej
na liście przed rozpoczęciem Męskiego Grania, oraz rapera O.S.T.R.,
który pomimo znacznego zaangażowania w Męskie Granie, na listę przebojów
dostał się jedynie z utworami współtworzonymi z Organkiem i Katarzyną
Nosowską. Należy pamiętać, że takich artystów z pewnością jest więcej.

Druga grupa jest o wiele bardziej interesująca. Możemy do niej zaliczyć
m.in. Brodkę, Katarzynę Nosowską, Lao Che czy Macieja Maleńczuka

    \begin{tcolorbox}[breakable, size=fbox, boxrule=1pt, pad at break*=1mm,colback=cellbackground, colframe=cellborder]
\prompt{In}{incolor}{204}{\boxspacing}
\begin{Verbatim}[commandchars=\\\{\}]
\PY{n}{druga\PYZus{}grupa} \PY{o}{\PYZlt{}\PYZhy{}} \PY{n+nf}{c}\PY{p}{(}\PY{l+s}{\PYZdq{}}\PY{l+s}{Nosowska\PYZdq{}}\PY{p}{,} \PY{l+s}{\PYZdq{}}\PY{l+s}{Brodka\PYZdq{}}\PY{p}{,} \PY{l+s}{\PYZdq{}}\PY{l+s}{Maleńczuk\PYZdq{}}\PY{p}{,} \PY{l+s}{\PYZdq{}}\PY{l+s}{Lao Che\PYZdq{}}\PY{p}{)}

\PY{n}{res} \PY{o}{\PYZpc{}\PYZgt{}\PYZpc{}} \PY{n+nf}{filter}\PY{p}{(}\PY{n}{Artysta} \PY{o}{\PYZpc{}in\PYZpc{}} \PY{n}{druga\PYZus{}grupa}\PY{p}{)} \PY{o}{\PYZpc{}\PYZgt{}\PYZpc{}}
\PY{n+nf}{ggplot}\PY{p}{(}\PY{p}{)} \PY{o}{+} \PY{n+nf}{ggtitle}\PY{p}{(}\PY{l+s}{\PYZdq{}}\PY{l+s}{Kumulowane punkty artystów \PYZbs{}\PYZdq{}Męskiego Grania \PYZbs{}\PYZdq{} \PYZhy{} \PYZbs{}\PYZdq{}Starzy wyjadacze\PYZbs{}\PYZdq{}\PYZdq{}}\PY{p}{)} \PY{o}{+}
    \PY{n+nf}{geom\PYZus{}line}\PY{p}{(}\PY{n+nf}{aes}\PY{p}{(}\PY{n}{Nr.notowania}\PY{p}{,} \PY{n}{KumulowanePunkty}\PY{p}{,} \PY{n}{color}\PY{o}{=}\PY{n}{Artysta}\PY{p}{)}\PY{p}{)} \PY{o}{+} \PY{n+nf}{theme\PYZus{}bw}\PY{p}{(}\PY{p}{)} \PY{o}{+} \PY{n+nf}{geom\PYZus{}vline}\PY{p}{(}\PY{n}{xintercept}\PY{o}{=}\PY{n}{nr\PYZus{}notowania\PYZus{}m\PYZus{}gran}\PY{p}{,} \PY{n}{linetype}\PY{o}{=}\PY{l+s}{\PYZdq{}}\PY{l+s}{dotted\PYZdq{}}\PY{p}{)}
\end{Verbatim}
\end{tcolorbox}

    \begin{center}
    \adjustimage{max size={0.9\linewidth}{0.9\paperheight}}{Analiza_Listy_Przebojow_Trojki_files/Analiza_Listy_Przebojow_Trojki_47_0.png}
    \end{center}
    { \hspace*{\fill} \\}
    
    O ile Katarzyna Nosowska (wokalistka grupy \emph{Hey}) radziła sobie
bardzo dobrze jeszcze przed współpracą z Męskim Graniem, tak Brodka i
Maleńczuk lepiej zaznaczyli swoje miejsce na liście po rozpoczęciu
omawianego festiwalu. Oczywiście nadinterpretacją byłoby stwierdzić, że
omawiani artyści zawdzięczają swoje wysokie pozycje jedynie festiwalowi
Żywca, ale z pewnością istnieje korelacja między ich popularnością na
liście a uczestnictwem w MG. *** Trzecia grupa to artyści, którzy
pojawili się na liście po rozpoczęciu Męskiego Grania. Są to m.in.
Organek, Dawid Podsiadło, Daria Zawiałow czy Krzysztow Zalewski

    \begin{tcolorbox}[breakable, size=fbox, boxrule=1pt, pad at break*=1mm,colback=cellbackground, colframe=cellborder]
\prompt{In}{incolor}{25}{\boxspacing}
\begin{Verbatim}[commandchars=\\\{\}]
\PY{n}{trzecia\PYZus{}grupa} \PY{o}{\PYZlt{}\PYZhy{}} \PY{n+nf}{c}\PY{p}{(}\PY{l+s}{\PYZdq{}}\PY{l+s}{Zawiałow\PYZdq{}}\PY{p}{,} \PY{l+s}{\PYZdq{}}\PY{l+s}{Organek\PYZdq{}}\PY{p}{,} \PY{l+s}{\PYZdq{}}\PY{l+s}{Podsiadło\PYZdq{}}\PY{p}{,} \PY{l+s}{\PYZdq{}}\PY{l+s}{Zalewski\PYZdq{}}\PY{p}{)}

\PY{n}{res} \PY{o}{\PYZpc{}\PYZgt{}\PYZpc{}} \PY{n+nf}{filter}\PY{p}{(}\PY{n}{Artysta} \PY{o}{\PYZpc{}in\PYZpc{}} \PY{n}{trzecia\PYZus{}grupa}\PY{p}{)} \PY{o}{\PYZpc{}\PYZgt{}\PYZpc{}}
\PY{n+nf}{ggplot}\PY{p}{(}\PY{p}{)} \PY{o}{+} 
\PY{n+nf}{geom\PYZus{}line}\PY{p}{(}\PY{n+nf}{aes}\PY{p}{(}\PY{n}{Nr.notowania}\PY{p}{,} \PY{n}{KumulowanePunkty}\PY{p}{,} \PY{n}{color}\PY{o}{=}\PY{n}{Artysta}\PY{p}{)}\PY{p}{)} \PY{o}{+} \PY{n+nf}{theme\PYZus{}bw}\PY{p}{(}\PY{p}{)} \PY{o}{+} \PY{n+nf}{geom\PYZus{}vline}\PY{p}{(}\PY{n}{xintercept}\PY{o}{=}\PY{n}{nr\PYZus{}notowania\PYZus{}m\PYZus{}gran}\PY{p}{,} \PY{n}{linetype}\PY{o}{=}\PY{l+s}{\PYZdq{}}\PY{l+s}{dotted\PYZdq{}}\PY{p}{)}
\end{Verbatim}
\end{tcolorbox}

    \begin{center}
    \adjustimage{max size={0.9\linewidth}{0.9\paperheight}}{Analiza_Listy_Przebojow_Trojki_files/Analiza_Listy_Przebojow_Trojki_49_0.png}
    \end{center}
    { \hspace*{\fill} \\}
    
    Należy zauważyć, że artyści nie dostawali wysokich miejsc tylko w
trakcie Męskiego Grania, lecz czasem niemal bez przerwy ``pięli się'' w
górę tabeli notowań. Trudno jest stwierdzić, czy popularność Męskiego
Grania pozwoliła im osiągnąć wysokie miejsca, czy może oni przynieśli
rozgłos festiwalowi. Dla przykładu Dawid Podsiadło w chwili rozpoczęcia
swojego pierwszego Męskiego Grania w 2014 miał już sumarycznie ok. 1700
punktów w Liście Przebojów, jednak do ostatniego notowania zdobył ich
jeszcze ponad 3-krotnie więcej.

    \hypertarget{podsumowanie}{%
\subsection{Podsumowanie}\label{podsumowanie}}

Lista przebojów jest ciekawym zbiorem danych. Powstaje przez długie lata
i może być świetnym przedmiotem analizy dla muzyków, producentów,
socjologów czy wreszcie - fanów. Chociaż większość z powyższych rozważań
to jedynie wstępne próby podsumowania i wyszukania zależności w długiej
i różnorodnej liście utworów, pokazuje, że we względnie losowym
przyporządkowaniu miejsca kilkudziesięciu utworom co tydzień przez
niemal 40 lat, można doszukać się prawidłowości.


    % Add a bibliography block to the postdoc
    
    
    
\end{document}
